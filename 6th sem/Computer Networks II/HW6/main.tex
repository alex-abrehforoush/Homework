\documentclass{article}

\usepackage{graphicx}
\usepackage{rotating}
\usepackage{amsmath}
\usepackage{fancyhdr}
\usepackage{listings}
\usepackage{xcolor}
\usepackage{color}
\usepackage{textcomp}
\usepackage{float}
\usepackage[sorting=none]{biblatex}
\usepackage[margin=1in]{geometry}
\usepackage[font={small,it}]{caption}
\usepackage{placeins}
\usepackage{xepersian}

%\DeclareMathOperator*{\btie}{\bowtie}
\addbibresource{bibliography.bib}
\settextfont[Scale=1.2]{B-NAZANIN.TTF}
\setlatintextfont[Scale=1]{Times New Roman}
\renewcommand{\baselinestretch}{1.5}
\pagestyle{fancy}
\fancyhf{}
\rhead{تکلیف ششم درس شبکه‌های کامپیوتری 2}
\lhead{\thepage}
\rfoot{علیرضا ابره فروش}
\lfoot{9816603}
\renewcommand{\headrulewidth}{1pt}
\renewcommand{\footrulewidth}{1pt}
%%%%%%%%%%
\lstset
{
    language=[latex]tex,
    basicstyle=\ttfamily,
    commentstyle=\color{black},
    columns=fullflexible,
    keepspaces=true,
    upquote=true,
    showstringspaces=false,
    morestring=[s]\\\%,
    stringstyle=\color{black},
}
%%%%%%%%%%
%beginMatlab
\definecolor{mygreen}{RGB}{28,172,0} % color values Red, Green, Blue
\definecolor{mylilas}{RGB}{170,55,241}
%endMatlab
\begin{document}
%beginMatlab
\lstset{language=Matlab,%
    %basicstyle=\color{red},
    breaklines=true,%
    morekeywords={matlab2tikz},
    keywordstyle=\color{blue},%
    morekeywords=[2]{1}, keywordstyle=[2]{\color{black}},
    identifierstyle=\color{black},%
    stringstyle=\color{mylilas},
    commentstyle=\color{mygreen},%
    showstringspaces=false,%without this there will be a symbol in the places where there is a space
    numbers=left,%
    numberstyle={\tiny \color{black}},% size of the numbers
    numbersep=9pt, % this defines how far the numbers are from the text
    emph=[1]{for,end,break},emphstyle=[1]\color{red}, %some words to emphasise
    %emph=[2]{word1,word2}, emphstyle=[2]{style},    
}
%endMatlab
\begin{titlepage}
\begin{center}
\includegraphics[width=0.4\textwidth]{IUT Logo.png}\\
        
\LARGE
\textbf{دانشگاه صنعتی اصفهان}\\
\textbf{دانشکده مهندسی برق و کامپیوتر}\\
        
\vfill
        
\huge
\textbf{عنوان: تکلیف اول درس سیستم‌های عامل 1}\\
        
\vfill
        
\LARGE
\textbf{نام و نام خانوادگی: علیرضا ابره فروش}\\
\textbf{شماره دانشجویی: 9816603}\\
\textbf{نیم\,سال تحصیلی: پاییز 1400}\\
\textbf{مدرّس: دکتر محمّدرضا حیدرپور}\\
\textbf{دستیاران آموزشی: مجید فرهادی - دانیال مهرآیین - محمّد نعیمی}\\
\end{center}
\end{titlepage}


%\tableofcontents
\newpage

\section{فصل 3}
\subsection{سوال 40}
\subsubsection{\lr{a}}
مکانیزم \lr{slowstart} در بازه‌های 
$[1,\: 6]$
و
$[23,\: 26]$
درحال اجرا است.

\subsubsection{\lr{b}}
مکانیزم \lr{congestion avoidance} در بازه‌های 
$[6,\: 16]$
و
$[17,\: 22]$
درحال اجرا است.
\subsubsection{\lr{c}}
از آنجایی که در 16امین \lr{transmission round}، \lr{Congestion window size} برابر 1 نشده است، پس نمی‌تواند \lr{timeout} رخ داده باشد. پس \lr{triple duplicate ACK} رخ داده است.
\subsubsection{\lr{d}}
از آنجایی که در 22امین \lr{transmission round}، \lr{Congestion window size} برابر 1 شده است، پس \lr{timeout} رخ داده است.
\subsubsection{\lr{e}}
\lr{ssthresh} 32 است. مکانیزم \lr{congestion avoidance} با توقف مکانیزم \lr{slowstart} شروع به کار کرده است.
\subsubsection{\lr{f}}
با رخ دادن \lr{packet loss} در 16امین \lr{transmission round} اندازه‌ی \lr{ssthresh} نصف می‌شود. پس در 18امین \lr{transmission round} اندازه‌ی \lr{ssthresh} برابر $42/2=21$ است.
\subsubsection{\lr{g}}
با رخ دادن \lr{packet loss} در 22امین \lr{transmission round} اندازه‌ی \lr{threshold} برابر نصف  \lr{Congestion window size} می‌شود. پس \lr{ssthresh} برابر 
$\left\lfloor \frac{29}{2} \right\rfloor = 14$
می‌شود.
\subsubsection{\lr{h}}
حین 1امین \lr{transmission round} پکت 1، حین 2امین \lr{transmission round} پکت‌های 2 و 3،  حین 3امین \lr{transmission round} پکت‌های 4 تا 7، حین 4امین \lr{transmission round} پکت‌های 8 تا 15، حین 5امین \lr{transmission round} پکت‌های 16 تا 31، حین 6امین \lr{transmission round} پکت‌های 32 تا 63، حین 7امین \lr{transmission round} پکت‌های 64 تا 96 ارسال می‌شوند که در نتیجه پکت 70 حین 7امین \lr{transmission round} ارسال می‌شود.
\subsubsection{\lr{i}}
\lr{threshold} برابر 4 که نصف اندازه‌ی فعلی پنجره می‌شود. همچنین اندازه‌ی پنجره برابر اندازه‌ی جدید \lr{threshold} به علاوه‌ی 3 می‌شود که برابر با 7 است.
\subsubsection{\lr{j}}
\lr{threshold} و \lr{congestion window size} به ترتیب برابر 21 و 1 می‌شود.
\subsubsection{\lr{k}}
$
1+2+4+8+16+21=52
$
پکت در \lr{transmission round}های 17 تا 22 ارسال شده است.


\subsection{سوال 46}
\subsubsection{\lr{a}}
فرض می‌کنیم \lr{W} بیشینه اندازه‌ی پنجره در مقایس سگمنت باشد. پس داریم:
$
\frac{W\times MSS}{RTT} = 10\:Mbps\Rightarrow W = \frac{10\times 10^6\times 0.15}{1500\times 8} = 125
$


\subsubsection{\lr{b}}
از آنجایی که اندازه‌ی \lr{congestion window} در رنج 
$
\frac{W}{2}
$
تا 
$
W
$
تغییر می‌کند، آنگاه متوسط اندازه‌ی پنجره برابر 
$
0.75W = \left\lceil 93.75 \right\rceil = 94
$
سگمنت است. متوسط \lr{throughput} برابر است با:
$
\frac{94\times1500\times8}{0.15}=7.52\:Mbps.
$

\subsubsection{\lr{c}}
با گم شدن بسته اندازه‌ی پنجره از $\frac{W}{2}$ به $W$ تغییر می‌کند. به عبارتی دیگر 
$
\left\lfloor \frac{125}{2} \right\rfloor=62
$
پس داریم:
$
(125 - 62)\times 0.15 = 9.45\:seconds
$
از آنجایی که تعداد \lr{RTT}های مورد نیاز \lr{TCP connection}ها برای افزایش اندازه‌ی پنجره از 62 به 125  برابر 63 است، در هر \lr{RTT} اندازه‌ی پنجره 1 واحد زیاد می‌شود.
%%%%%%%%%%%%%%%%%%%%%%%%%%%%%%%%%%%
%%%%%%%%%%%%%%%%%%%%%%%%%%%%%%%%%%%
%%%%%%%%%%%%%%%%%%%%%%%%%%%%%%%%%%%

\section*{منابع}
\renewcommand{\section}[2]{}%
\begin{thebibliography}{99} % assumes less than 100 references
%چنانچه مرجع فارسی نیز داشته باشید باید دستور فوق را فعال کنید و مراجع فارسی خود را بعد از این دستور وارد کنید


\begin{LTRitems}

\resetlatinfont

\bibitem{b1}
\end{LTRitems}

\end{thebibliography}


\end{document}

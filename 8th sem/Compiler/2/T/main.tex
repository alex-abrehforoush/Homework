\documentclass{article}

\usepackage{graphicx}
\usepackage{rotating}
\usepackage{amsmath}
\usepackage{amssymb}
\usepackage{fancyhdr}
\usepackage{listings}
\usepackage{lscape}
%\usepackage{xcolor}
\usepackage{color}
\usepackage{amsfonts}
\usepackage{textcomp}
\usepackage{float}
\usepackage{longtable}
\usepackage{booktabs}
\usepackage[sorting=none]{biblatex}
\usepackage[margin=1in]{geometry}
\usepackage[font={small,it}]{caption}
\usepackage[table,xcdraw]{xcolor}
\usepackage{placeins}
\usepackage{xepersian}





%\DeclareMathOperator*{\btie}{\bowtie}
\addbibresource{bibliography.bib}
\settextfont[Scale=1.2]{B-NAZANIN.TTF}
\setlatintextfont[Scale=1]{Times New Roman}
\renewcommand{\baselinestretch}{1.5}
\pagestyle{fancy}
\fancyhf{}
\rhead{تکلیف تئوری دوم درس کامپایلر}
\lhead{\thepage}
\rfoot{علیرضا ابره فروش}
\lfoot{9816603}
\renewcommand{\headrulewidth}{1pt}
\renewcommand{\footrulewidth}{1pt}
%%%%%%%%%%
\lstset
{
    language=[latex]tex,
    basicstyle=\ttfamily,
    commentstyle=\color{black},
    columns=fullflexible,
    keepspaces=true,
    upquote=true,
    showstringspaces=false,
    morestring=[s]\\\%,
    stringstyle=\color{black},
}
%%%%%%%%%%
%beginMatlab
\definecolor{mygreen}{RGB}{28,172,0} % color values Red, Green, Blue
\definecolor{mylilas}{RGB}{170,55,241}
%endMatlab
\begin{document}
%beginMatlab
\lstset{language=Matlab,%
    %basicstyle=\color{red},
    breaklines=true,%
    morekeywords={matlab2tikz},
    keywordstyle=\color{blue},%
    morekeywords=[2]{1}, keywordstyle=[2]{\color{black}},
    identifierstyle=\color{black},%
    stringstyle=\color{mylilas},
    commentstyle=\color{mygreen},%
    showstringspaces=false,%without this there will be a symbol in the places where there is a space
    numbers=left,%
    numberstyle={\tiny \color{black}},% size of the numbers
    numbersep=9pt, % this defines how far the numbers are from the text
    emph=[1]{for,end,break},emphstyle=[1]\color{red}, %some words to emphasise
    %emph=[2]{word1,word2}, emphstyle=[2]{style},    
}
%endMatlab
\begin{titlepage}
\begin{center}
\includegraphics[width=0.4\textwidth]{IUT Logo.png}\\
        
\LARGE
\textbf{دانشگاه صنعتی اصفهان}\\
\textbf{دانشکده مهندسی برق و کامپیوتر}\\
        
\vfill
        
\huge
\textbf{عنوان: تکلیف اول درس سیستم‌های عامل 1}\\
        
\vfill
        
\LARGE
\textbf{نام و نام خانوادگی: علیرضا ابره فروش}\\
\textbf{شماره دانشجویی: 9816603}\\
\textbf{نیم\,سال تحصیلی: پاییز 1400}\\
\textbf{مدرّس: دکتر محمّدرضا حیدرپور}\\
\textbf{دستیاران آموزشی: مجید فرهادی - دانیال مهرآیین - محمّد نعیمی}\\
\end{center}
\end{titlepage}


%\tableofcontents
\newpage

\section{}%1
\subsection{\lr{a}}
طبق الگوریتم زیر عمل می‌کنیم.
\begin{figure}[H]
    \centering
    \includegraphics[width=0.75\textwidth]{figures/1a.png}
    \caption
	{}
    \label{fig:fig1}
\end{figure}

\begin{latin}
$
S \longrightarrow SS ^ \prime \\
S ^ {\prime} \longrightarrow +S|+P \\
P \longrightarrow PP ^ \prime \\
P ^ \prime \longrightarrow *P|*I \\
I \longrightarrow -I|(S)|D \\
D \longrightarrow 0|1N \\
N \longrightarrow NN ^ \prime|0|1|\varepsilon \\
N ^ \prime \longrightarrow N
$
\end{latin}

\subsection{\lr{b}}
طبق الگوریتم زیر عمل می‌کنیم.
\begin{figure}[H]
    \centering
    \includegraphics[width=0.75\textwidth]{figures/1b.png}
    \caption
	{}
    \label{fig:fig1}
\end{figure}

\begin{latin}
$
S \longrightarrow US ^ \prime \\
S ^ \prime \longrightarrow aSS ^ \prime | \varepsilon \\
U \longrightarrow TU ^ \prime \\
U ^ \prime \longrightarrow uUU ^ \prime | \varepsilon \\
T \longrightarrow tT ^ \prime | fT ^ \prime | (S)T ^ \prime \\
T ^ \prime \longrightarrow | nT ^ \prime | \varepsilon
$
\end{latin}

\section{}%2

\section{}%3
\subsection{\lr{a}}
تصویر زیر بیان شرط لازم و کافی برای \lr{LL(1)} بودن یک گرامر را شرح می‌دهد.
\begin{figure}[H]
    \centering
    \includegraphics[width=0.75\textwidth]{figures/3a.png}
    \caption
	{}
    \label{fig:fig1}
\end{figure}
با توجه به این قضیه، داریم:

\begin{latin}
$
FIRST(Z) = \lbrace b, \varepsilon \rbrace \\
FIRST(Y) = \lbrace b, c \rbrace \\
FIRST(bX) = \lbrace b \rbrace \\
\Rightarrow FIRST(bX) \bigcap FIRST(Y) \neq \emptyset \\
, \\
FIRST(bZ) = \lbrace b \rbrace \\
FOLLOW(Z) = \lbrace c \rbrace \\
\Rightarrow FIRST(bZ) \bigcap FOLLOW(Z) = \emptyset
$
\end{latin}
از آنجایی که $FIRST(bX) \bigcap FIRST(Y) \neq \emptyset$ پس گرامر \lr{LL(1)} نیست.

\subsection{\lr{b}}
با حذف \lr{production}ِ $X \longrightarrow bX$ گرامر \lr{LL(1)} می‌شود. چون بین دو شرط بالا اولی که با حذف $bX$ ارضا می‌شود و دومی هم برقرار است.


\section{}%4

\subsection{\lr{a}}
\begin{latin}
% Please add the following required packages to your document preamble:
% \usepackage{graphicx}
\begin{table}[H]
\centering
\resizebox{\textwidth}{!}{%
\begin{tabular}{|c|c|c|}
\hline
              & FIRST                                              & FOLLOW                           \\ \hline
S             & $\left\{ print, \textbf{ID}, \varepsilon \right\}$ & \{ \$ \}                         \\ \hline
ComponentList & $\left\{ print, \textbf{ID}, \varepsilon \right\}$ & \{ \$ \}                         \\ \hline
Component     & $\left\{ print, \textbf{ID} \right\}$              & $\left\{ ; \right\}$             \\ \hline
Expression    & $\left\{ (, \textbf{ID}, \textbf{NUM} \right\}$    & $\left\{ ), ; \right\}$          \\ \hline
Operator      & $\left\{ \textbf{ID}, \textbf{NUM} \right\}$       & $\left\{ +, -, *, ), ; \right\}$ \\ \hline
NextStage     & $\left\{ +, -, *, \varepsilon \right\}$            & $\left\{ ), ; \right\}$          \\ \hline
Operation     & $\left\{ +, -, * \right\}$                         & $\left\{ ), ; \right\}$          \\ \hline
\end{tabular}%
}
\end{table}
\end{latin}






\subsection{\lr{b}}

\begin{latin}
% Please add the following required packages to your document preamble:
% \usepackage{graphicx}
\begin{table}[H]
\centering
\resizebox{\textwidth}{!}{%
\begin{tabular}{|c|c|c|c|c|c|c|c|c|c|c|c|}
\hline
              & ;                                       & print                                                     & (                                         & )                                       & \textbf{ID}                              & = & \textbf{NUM}                  & +                                       & -                                       & *                                       & \$                                          \\ \hline
S             &                                         & $S \longrightarrow ComponentList$                         &                                           &                                         & $S \longrightarrow ComponentList$                         &   &                                                &                                         &                                         &                                         & $S \longrightarrow ComponentList$           \\ \hline
ComponentList &                                         & $ComponentList \longrightarrow Component ; ComponentList$ &                                           &                                         & $ComponentList \longrightarrow Component ; ComponentList$ &   &                                                &                                         &                                         &                                         & $ComponentList \longrightarrow \varepsilon$ \\ \hline
Component     &                                         & $Component \longrightarrow print(Expression)$             &                                           &                                         & $Component \longrightarrow \textbf{ID} = Expression$      &   &                                                &                                         &                                         &                                         &                                             \\ \hline
Expression    &                                         &                                                           & $Expression \longrightarrow (Expression)$ &                                         & $Expression \longrightarrow Operand NextStage$            &   & $Expression \longrightarrow Operand NextStage$ &                                         &                                         &                                         &                                             \\ \hline
Operand       &                                         &                                                           &                                           &                                         & $Operand \longrightarrow \textbf{ID}$                     &   & $Operand \longrightarrow \textbf{NUM}$         &                                         &                                         &                                         &                                             \\ \hline
NextStage     & $NextStage \longrightarrow \varepsilon$ &                                                           &                                           & $NextStage \longrightarrow \varepsilon$ &                                                           &   &                                                & $NextStage \longrightarrow Operation$   & $NextStage \longrightarrow Operation$   & $NextStage \longrightarrow Operation$   &                                             \\ \hline
Operation     &                                         &                                                           &                                           &                                         &                                                           &   &                                                & $Operation \longrightarrow +Expression$ & $Operation \longrightarrow -Expression$ & $Operation \longrightarrow *Expression$ &                                             \\ \hline
\end{tabular}%
}
\end{table}
\end{latin}


\subsection{\lr{c}}
\begin{latin}
% Please add the following required packages to your document preamble:
% \usepackage{graphicx}
\begin{table}[H]
\resizebox{\columnwidth}{!}{%
\begin{tabular}{|l|r|r|l|}
\hline
\multicolumn{1}{|c|}{\textbf{Matched}}                                                & \multicolumn{1}{c|}{\textbf{Stack}}         & \multicolumn{1}{c|}{\textbf{Input}}                                                      & \multicolumn{1}{c|}{\textbf{Action}}                            \\ \hline
                                                                                      & S \$                                        & \textbf{ID} = \textbf{NUM} + ( \textbf{NUM} * \textbf{ID} ) ; print ( \textbf{ID} ) ; \$ &                                                                 \\ \hline
                                                                                      & ComponentList \$                            & \textbf{ID} = \textbf{NUM} + ( \textbf{NUM} * \textbf{ID} ) ; print ( \textbf{ID} ) ; \$ & output S $\longrightarrow$ ComponentList                        \\ \hline
                                                                                      & Component ; ComponentList \$                & \textbf{ID} = \textbf{NUM} + ( \textbf{NUM} * \textbf{ID} ) ; print ( \textbf{ID} ) ; \$ & output ComponentList $longrightarrow$ Component ; ComponentList \\ \hline
                                                                                      & \textbf{ID} = Expression ; ComponentList \$ & \textbf{ID} = \textbf{NUM} + ( \textbf{NUM} * \textbf{ID} ) ; print ( \textbf{ID} ) ; \$ & output Component $longrightarrow$ \textbf{ID} = Expression      \\ \hline
\textbf{ID}                                                                           & = Expression ; ComponentList \$             & = \textbf{NUM} + ( \textbf{NUM} * \textbf{ID} ) ; print ( \textbf{ID} ) ; \$             & match \textbf{ID}                                               \\ \hline
\textbf{ID} =                                                                         & Expression ; ComponentList \$               & \textbf{NUM} + ( \textbf{NUM} * \textbf{ID} ) ; print ( \textbf{ID} ) ; \$               & match =                                                         \\ \hline
\textbf{ID} =                                                                         & Operand NextStage ; ComponentList \$        & \textbf{NUM} + ( \textbf{NUM} * \textbf{ID} ) ; print ( \textbf{ID} ) ; \$               & output Expression $longrightarrow$ Operand NextStage            \\ \hline
\textbf{ID} =                                                                         & \textbf{NUM} NextStage ; ComponentList \$   & \textbf{NUM} + ( \textbf{NUM} * \textbf{ID} ) ; print ( \textbf{ID} ) ; \$               & output Operand $longrightarrow$ \textbf{NUM}                    \\ \hline
\textbf{ID} = \textbf{NUM}                                                            & NextStage ; ComponentList \$                & + ( \textbf{NUM} * \textbf{ID} ) ; print ( \textbf{ID} ) ; \$                            & match \textbf{NUM}                                              \\ \hline
\textbf{ID} = \textbf{NUM}                                                            & Operation ; ComponentList \$                & + ( \textbf{NUM} * \textbf{ID} ) ; print ( \textbf{ID} ) ; \$                            & output NextStage $longrightarrow$ Operation                     \\ \hline
\textbf{ID} = \textbf{NUM}                                                            & + Expression ; ComponentList \$             & + ( \textbf{NUM} * \textbf{ID} ) ; print ( \textbf{ID} ) ; \$                            & output Operation $longrightarrow$ + Expression                  \\ \hline
\textbf{ID} = \textbf{NUM} +                                                          & Expression ; ComponentList \$               & ( \textbf{NUM} * \textbf{ID} ) ; print ( \textbf{ID} ) ; \$                              & match +                                                         \\ \hline
\textbf{ID} = \textbf{NUM} +                                                          & ( Expression ) ; ComponentList \$           & ( \textbf{NUM} * \textbf{ID} ) ; print ( \textbf{ID} ) ; \$                              & output Expression $longrightarrow$ ( Expression )               \\ \hline
\textbf{ID} = \textbf{NUM} + (                                                        & Expression ) ; ComponentList \$             & \textbf{NUM} * \textbf{ID} ) ; print ( \textbf{ID} ) ; \$                                & match (                                                         \\ \hline
\textbf{ID} = \textbf{NUM} + (                                                        & Operand NextStage ) ; ComponentList \$      & \textbf{NUM} * \textbf{ID} ) ; print ( \textbf{ID} ) ; \$                                & output Expression $longrightarrow$ Operand NextStage            \\ \hline
\textbf{ID} = \textbf{NUM} + (                                                        & \textbf{NUM} NextStage ) ; ComponentList \$ & \textbf{NUM} * \textbf{ID} ) ; print ( \textbf{ID} ) ; \$                                & output Operand $longrightarrow$ \textbf{NUM}                    \\ \hline
\textbf{ID} = \textbf{NUM} + ( \textbf{NUM}                                           & NextStage ) ; ComponentList \$              & * \textbf{ID} ) ; print ( \textbf{ID} ) ; \$                                             & match \textbf{NUM}                                              \\ \hline
\textbf{ID} = \textbf{NUM} + ( \textbf{NUM}                                           & Operation ) ; ComponentList \$              & * \textbf{ID} ) ; print ( \textbf{ID} ) ; \$                                             & output NextStage $longrightarrow$ Operation                     \\ \hline
\textbf{ID} = \textbf{NUM} + ( \textbf{NUM}                                           & * Expression ) ; ComponentList \$           & * \textbf{ID} ) ; print ( \textbf{ID} ) ; \$                                             & output Operation $longrightarrow$ * Expression                  \\ \hline
\textbf{ID} = \textbf{NUM} + ( \textbf{NUM} *                                         & Expression ) ; ComponentList \$             & \textbf{ID} ) ; print ( \textbf{ID} ) ; \$                                               & match *                                                         \\ \hline
\textbf{ID} = \textbf{NUM} + ( \textbf{NUM} *                                         & Operand NextStage ) ; ComponentList \$      & \textbf{ID} ) ; print ( \textbf{ID} ) ; \$                                               & output Expression $longrightarrow$ Operand NextStage            \\ \hline
\textbf{ID} = \textbf{NUM} + ( \textbf{NUM} *                                         & \textbf{ID} NextStage ) ; ComponentList \$  & \textbf{ID} ) ; print ( \textbf{ID} ) ; \$                                               & output Operand $longrightarrow$ \textbf{ID}                     \\ \hline
\textbf{ID} = \textbf{NUM} + ( \textbf{NUM} * \textbf{ID}                             & NextStage ) ; ComponentList \$              & ) ; print ( \textbf{ID} ) ; \$                                                           & match \textbf{ID}                                               \\ \hline
\textbf{ID} = \textbf{NUM} + ( \textbf{NUM} * \textbf{ID}                             & ) ; ComponentList \$                        & ) ; print ( \textbf{ID} ) ; \$                                                           & output NextStage $longrightarrow \varepsilon$                   \\ \hline
\textbf{ID} = \textbf{NUM} + ( \textbf{NUM} * \textbf{ID} )                           & ; ComponentList \$                          & ; print ( \textbf{ID} ) ; \$                                                             & match )                                                         \\ \hline
\textbf{ID} = \textbf{NUM} + ( \textbf{NUM} * \textbf{ID} ) ;                         & ComponentList \$                            & print ( \textbf{ID} ) ; \$                                                               & match ;                                                         \\ \hline
\textbf{ID} = \textbf{NUM} + ( \textbf{NUM} * \textbf{ID} ) ;                         & Component ; ComponentList \$                & print ( \textbf{ID} ) ; \$                                                               & output ComponentList $longrightarrow$ Component ; ComponentList \\ \hline
\textbf{ID} = \textbf{NUM} + ( \textbf{NUM} * \textbf{ID} ) ;                         & print ( Experssion ) ; ComponentList \$     & print ( \textbf{ID} ) ; \$                                                               & output Component $longrightarrow$ print ( Expression )          \\ \hline
\textbf{ID} = \textbf{NUM} + ( \textbf{NUM} * \textbf{ID} ) ; print                   & ( Experssion ) ; ComponentList \$           & ( \textbf{ID} ) ; \$                                                                     & match print                                                     \\ \hline
\textbf{ID} = \textbf{NUM} + ( \textbf{NUM} * \textbf{ID} ) ; print (                 & Experssion ) ; ComponentList \$             & \textbf{ID} ) ; \$                                                                       & match (                                                         \\ \hline
\textbf{ID} = \textbf{NUM} + ( \textbf{NUM} * \textbf{ID} ) ; print (                 & Operand NextStage ) ; ComponentList \$      & \textbf{ID} ) ; \$                                                                       & output Expression $longrightarrow$ Operand NextStage            \\ \hline
\textbf{ID} = \textbf{NUM} + ( \textbf{NUM} * \textbf{ID} ) ; print (                 & \textbf{ID} NextStage ) ; ComponentList \$  & \textbf{ID} ) ; \$                                                                       & output Operand $longrightarrow$ \textbf{ID}                     \\ \hline
\textbf{ID} = \textbf{NUM} + ( \textbf{NUM} * \textbf{ID} ) ; print ( \textbf{ID}     & NextStage ) ; ComponentList \$              & ) ; \$                                                                                   & match \textbf{ID}                                               \\ \hline
\textbf{ID} = \textbf{NUM} + ( \textbf{NUM} * \textbf{ID} ) ; print ( \textbf{ID}     & ) ; ComponentList \$                        & ) ; \$                                                                                   & output NextStage $longrightarrow \varepsilon$                   \\ \hline
\textbf{ID} = \textbf{NUM} + ( \textbf{NUM} * \textbf{ID} ) ; print ( \textbf{ID} )   & ; ComponentList \$                          & ; \$                                                                                     & match )                                                         \\ \hline
\textbf{ID} = \textbf{NUM} + ( \textbf{NUM} * \textbf{ID} ) ; print ( \textbf{ID} ) ; & ComponentList \$                            & \$                                                                                       & match ;                                                         \\ \hline
\textbf{ID} = \textbf{NUM} + ( \textbf{NUM} * \textbf{ID} ) ; print ( \textbf{ID} ) ; & \$                                          & \$                                                                                       & output ComponentList $longrightarrow \varepsilon$               \\ \hline
\end{tabular}%
}
\end{table}
\end{latin}






%%%%%%%%%%%%%%%


\section{}
\subsection{}
خطاهای شناسایی شده توسط تحلیلگر لغوی معمولا دارای ویژگی های زیر هستند:
\begin{itemize}
\item این خطاها زمانی رخ می‌دهند که دنباله ورودی از کاراکترهای معتبر در زبان برنامه‌نویسی مورد نظر تشخیص داده نمی‌شود.
\item این خطاها به طور کلی توسط تحلیلگر لغوی تشخیص داده می‌شوند که در مرحله اول فرایند کامپایل است.
\item این خطاها اغلب به دلیل خطاهای نحوی، مانند کلمات کلیدی یا شناسه‌های نوشتاری نادرست یا کاراکترهای نامعتبر یا نمادها ایجاد می‌شوند.
\end{itemize}

\subsection{}
چهار نمونه مختلف از انواع خطاهای شناسایی شده توسط تحلیل‌گر لغوی، شامل موارد زیر می‌شوند:

\begin{enumerate}
\item کاراکترهای غیرمجاز: این کاراکترها که در زبان برنامه‌نویسی شناخته نشده‌اند، مانند کاراکترهای غیر چاپ‌پذیر یا نویسه‌هایی از زبان‌های دیگر هستند.
\item عدم تطابق نقل قول: این خطا هنگامی رخ می‌دهد که یک متن رشته‌ای به درستی با نقل قول متناظر خاتمه نمی‌یابد. به عنوان مثال، \lr{"Hello، World!} احتمالاً به دلیل عدم وجود دابل کوتیشن پایانی، با خطای لغوی روبرو می‌شود.
\item
\end{enumerate}












%%%%%%%%%%%%%%%%%%%%%%%%%%%%%%%%%%%%%%%%%%%%%%%%%%%%%%%%%%%%%%%%%%%%%%

%\begin{latin}
%\lstinputlisting{sources/p2.m}
%\end{latin}


%%%%%%%%%%%%%%%%%%%%%%%%%%%%%%%%%%%
%%%%%%%%%%%%%%%%%%%%%%%%%%%%%%%%%%%
%%%%%%%%%%%%%%%%%%%%%%%%%%%%%%%%%%%

%------------------------------------------------------------------------------------------


\subsection*{منابع}
\renewcommand{\subsection}[2]{}%
\begin{thebibliography}{99} % assumes less than 100 references
%چنانچه مرجع فارسی نیز داشته باشید باید دستور فوق را فعال کنید و مراجع فارسی خود را بعد از این دستور وارد کنید


\begin{LTRitems}

\resetlatinfont

\bibitem{b1}
\end{LTRitems}

\end{thebibliography}


\end{document}

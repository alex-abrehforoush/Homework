\documentclass{article}

\usepackage{graphicx}
\usepackage{rotating}
\usepackage{amsmath}
\usepackage{amssymb}
\usepackage{fancyhdr}
\usepackage{listings}
%\usepackage{xcolor}
\usepackage{color}
\usepackage{amsfonts}
\usepackage{textcomp}
\usepackage{float}
\usepackage[sorting=none]{biblatex}
\usepackage[margin=1in]{geometry}
\usepackage[font={small,it}]{caption}
\usepackage[table,xcdraw]{xcolor}
\usepackage{placeins}
\usepackage{xepersian}





%\DeclareMathOperator*{\btie}{\bowtie}
\addbibresource{bibliography.bib}
\settextfont[Scale=1.2]{B-NAZANIN.TTF}
\setlatintextfont[Scale=1]{Times New Roman}
\renewcommand{\baselinestretch}{1.5}
\pagestyle{fancy}
\fancyhf{}
\rhead{تکلیف اول درس مبانی هوش محاسباتی}
\lhead{\thepage}
\rfoot{علیرضا ابره فروش}
\lfoot{9816603}
\renewcommand{\headrulewidth}{1pt}
\renewcommand{\footrulewidth}{1pt}
%%%%%%%%%%
\lstset
{
    language=[latex]tex,
    basicstyle=\ttfamily,
    commentstyle=\color{black},
    columns=fullflexible,
    keepspaces=true,
    upquote=true,
    showstringspaces=false,
    morestring=[s]\\\%,
    stringstyle=\color{black},
}
%%%%%%%%%%
%beginMatlab
\definecolor{mygreen}{RGB}{28,172,0} % color values Red, Green, Blue
\definecolor{mylilas}{RGB}{170,55,241}
%endMatlab
\begin{document}
%beginMatlab
\lstset{language=Matlab,%
    %basicstyle=\color{red},
    breaklines=true,%
    morekeywords={matlab2tikz},
    keywordstyle=\color{blue},%
    morekeywords=[2]{1}, keywordstyle=[2]{\color{black}},
    identifierstyle=\color{black},%
    stringstyle=\color{mylilas},
    commentstyle=\color{mygreen},%
    showstringspaces=false,%without this there will be a symbol in the places where there is a space
    numbers=left,%
    numberstyle={\tiny \color{black}},% size of the numbers
    numbersep=9pt, % this defines how far the numbers are from the text
    emph=[1]{for,end,break},emphstyle=[1]\color{red}, %some words to emphasise
    %emph=[2]{word1,word2}, emphstyle=[2]{style},    
}
%endMatlab
\begin{titlepage}
\begin{center}
\includegraphics[width=0.4\textwidth]{IUT Logo.png}\\
        
\LARGE
\textbf{دانشگاه صنعتی اصفهان}\\
\textbf{دانشکده مهندسی برق و کامپیوتر}\\
        
\vfill
        
\huge
\textbf{عنوان: تکلیف اول درس سیستم‌های عامل 1}\\
        
\vfill
        
\LARGE
\textbf{نام و نام خانوادگی: علیرضا ابره فروش}\\
\textbf{شماره دانشجویی: 9816603}\\
\textbf{نیم\,سال تحصیلی: پاییز 1400}\\
\textbf{مدرّس: دکتر محمّدرضا حیدرپور}\\
\textbf{دستیاران آموزشی: مجید فرهادی - دانیال مهرآیین - محمّد نعیمی}\\
\end{center}
\end{titlepage}


%\tableofcontents
\newpage


\section{}%1
\subsection{نتیجه (خروجی)}
\begin{latin}
A linear regression model relies on a continuous dependent variable. This implies that the dependent variable takes up numeric values instead of being classified under categories or groups. In contrast, logistic regression models rely on binary dependent variables. The dependent (or response) variable can take up only two values – 0 or 1.
Also, linear regression output has a continuous value (it gives a range of values). For example,
\begin{itemize}
	\item Length of the roof (25 inches, 19 inches, 5 ft)
	\item Height (5 ft 8 inches, 6 ft 2 inches, 5 ft 10 inches)
	\item Escape velocity (26000 mph, 21500 mph, 29500 mph)
\end{itemize}
On the other hand, the logistic regression model is revealed via probabilities. For example,
\begin{itemize}
	\item 84.3\% chance of losing a tennis match
	\item 23.1\% chance of passing a bill in Congress
	\item 65.1\% chance of imposing a curfew during a COVID-19 outbreak
\end{itemize}
Moreover, linear regression observes a normal or gaussian distribution, and logistic regression reveals a binomial distribution.
\end{latin}

\subsection{ارتباط بین متغیرها}
\begin{latin}
Understanding the relationship between variables is crucial when deciding the type of regression model to be used for different purposes.

Linear regression describes a linear relationship between variables by plotting a straight line on a graph. It enables professionals to check on these linear relationships and track their movement over a period. On the contrary, logistic regression is known to study and examine the probability of an event occurrence. Since it does not denote a linear structure of a variable relationship, tracking logistic regression using linear structures is not required.
\end{latin}

\subsection{خطا}
\begin{latin}

\end{latin}








\section{}%1
\subsection{سوال 1.7}
\begin{latin}
% Please add the following required packages to your document preamble:
% \usepackage{graphicx}
\begin{table}[H]
\centering
\begin{tabular}{ccccc}
+                      & 0                      & 1                      & 2                      & 3                      \\ \cline{2-5} 
\multicolumn{1}{c|}{0} & \multicolumn{1}{c|}{0} & \multicolumn{1}{c|}{1} & \multicolumn{1}{c|}{2} & \multicolumn{1}{c|}{3} \\ \cline{2-5} 
\multicolumn{1}{c|}{1} & \multicolumn{1}{c|}{1} & \multicolumn{1}{c|}{2} & \multicolumn{1}{c|}{3} & \multicolumn{1}{c|}{0} \\ \cline{2-5} 
\multicolumn{1}{c|}{2} & \multicolumn{1}{c|}{2} & \multicolumn{1}{c|}{3} & \multicolumn{1}{c|}{0} & \multicolumn{1}{c|}{1} \\ \cline{2-5} 
\multicolumn{1}{c|}{3} & \multicolumn{1}{c|}{3} & \multicolumn{1}{c|}{0} & \multicolumn{1}{c|}{1} & \multicolumn{1}{c|}{2} \\ \cline{2-5} 
\end{tabular}
\end{table}
\end{latin}

\subsubsection{سوال 1.7.1}
\begin{latin}
% Please add the following required packages to your document preamble:
% \usepackage{graphicx}
\begin{table}[H]
\centering
\begin{tabular}{ccccc}
$\times$               & 0                      & 1                      & 2                      & 3                      \\ \cline{2-5} 
\multicolumn{1}{c|}{0} & \multicolumn{1}{c|}{0} & \multicolumn{1}{c|}{0} & \multicolumn{1}{c|}{0} & \multicolumn{1}{c|}{0} \\ \cline{2-5} 
\multicolumn{1}{c|}{1} & \multicolumn{1}{c|}{0} & \multicolumn{1}{c|}{1} & \multicolumn{1}{c|}{2} & \multicolumn{1}{c|}{3} \\ \cline{2-5} 
\multicolumn{1}{c|}{2} & \multicolumn{1}{c|}{0} & \multicolumn{1}{c|}{2} & \multicolumn{1}{c|}{0} & \multicolumn{1}{c|}{2} \\ \cline{2-5} 
\multicolumn{1}{c|}{3} & \multicolumn{1}{c|}{0} & \multicolumn{1}{c|}{3} & \multicolumn{1}{c|}{2} & \multicolumn{1}{c|}{1} \\ \cline{2-5} 
\end{tabular}
\end{table}
\end{latin}

\subsubsection{سوال 1.7.2}
\begin{latin}
% Please add the following required packages to your document preamble:
% \usepackage{graphicx}
\begin{table}[H]
\centering
\begin{tabular}{cccccc}
+                      & 0                      & 1                      & 2                      & 3                      & \multicolumn{1}{l}{4}  \\ \cline{2-6} 
\multicolumn{1}{c|}{0} & \multicolumn{1}{c|}{0} & \multicolumn{1}{c|}{1} & \multicolumn{1}{c|}{2} & \multicolumn{1}{c|}{3} & \multicolumn{1}{l|}{4} \\ \cline{2-6} 
\multicolumn{1}{c|}{1} & \multicolumn{1}{c|}{1} & \multicolumn{1}{c|}{2} & \multicolumn{1}{c|}{3} & \multicolumn{1}{c|}{4} & \multicolumn{1}{c|}{0} \\ \cline{2-6} 
\multicolumn{1}{c|}{2} & \multicolumn{1}{c|}{2} & \multicolumn{1}{c|}{3} & \multicolumn{1}{c|}{4} & \multicolumn{1}{c|}{0} & \multicolumn{1}{c|}{1} \\ \cline{2-6} 
\multicolumn{1}{c|}{3} & \multicolumn{1}{c|}{3} & \multicolumn{1}{c|}{4} & \multicolumn{1}{c|}{0} & \multicolumn{1}{c|}{1} & \multicolumn{1}{c|}{2} \\ \cline{2-6} 
\multicolumn{1}{l|}{4} & \multicolumn{1}{c|}{4} & \multicolumn{1}{c|}{0} & \multicolumn{1}{c|}{1} & \multicolumn{1}{c|}{2} & \multicolumn{1}{c|}{3} \\ \cline{2-6} 
\end{tabular}
\end{table}
\end{latin}

\begin{latin}
% Please add the following required packages to your document preamble:
% \usepackage{graphicx}
\begin{table}[H]
\centering
\begin{tabular}{cccccc}
$\times$               & 0                      & 1                      & 2                      & 3                      & 4                      \\ \cline{2-6} 
\multicolumn{1}{c|}{0} & \multicolumn{1}{c|}{0} & \multicolumn{1}{c|}{0} & \multicolumn{1}{c|}{0} & \multicolumn{1}{c|}{0} & \multicolumn{1}{c|}{0} \\ \cline{2-6} 
\multicolumn{1}{c|}{1} & \multicolumn{1}{c|}{0} & \multicolumn{1}{c|}{1} & \multicolumn{1}{c|}{2} & \multicolumn{1}{c|}{3} & \multicolumn{1}{c|}{4} \\ \cline{2-6} 
\multicolumn{1}{c|}{2} & \multicolumn{1}{c|}{0} & \multicolumn{1}{c|}{2} & \multicolumn{1}{c|}{4} & \multicolumn{1}{c|}{1} & \multicolumn{1}{c|}{3} \\ \cline{2-6} 
\multicolumn{1}{c|}{3} & \multicolumn{1}{c|}{0} & \multicolumn{1}{c|}{3} & \multicolumn{1}{c|}{1} & \multicolumn{1}{c|}{4} & \multicolumn{1}{c|}{2} \\ \cline{2-6} 
\multicolumn{1}{c|}{4} & \multicolumn{1}{c|}{0} & \multicolumn{1}{c|}{4} & \multicolumn{1}{c|}{3} & \multicolumn{1}{c|}{2} & \multicolumn{1}{c|}{1} \\ \cline{2-6} 
\end{tabular}
\end{table}
\end{latin}


\subsubsection{سوال 1.7.3}
\begin{latin}
% Please add the following required packages to your document preamble:
% \usepackage{graphicx}
\begin{table}[H]
\centering
\begin{tabular}{ccccccc}
+                      & 0                      & 1                      & 2                      & 3                      & 4                      & 5                      \\ \cline{2-7} 
\multicolumn{1}{c|}{0} & \multicolumn{1}{c|}{0} & \multicolumn{1}{c|}{1} & \multicolumn{1}{c|}{2} & \multicolumn{1}{c|}{3} & \multicolumn{1}{c|}{4} & \multicolumn{1}{c|}{5} \\ \cline{2-7} 
\multicolumn{1}{c|}{1} & \multicolumn{1}{c|}{1} & \multicolumn{1}{c|}{2} & \multicolumn{1}{c|}{3} & \multicolumn{1}{c|}{4} & \multicolumn{1}{c|}{5} & \multicolumn{1}{c|}{0} \\ \cline{2-7} 
\multicolumn{1}{c|}{2} & \multicolumn{1}{c|}{2} & \multicolumn{1}{c|}{3} & \multicolumn{1}{c|}{4} & \multicolumn{1}{c|}{5} & \multicolumn{1}{c|}{0} & \multicolumn{1}{c|}{1} \\ \cline{2-7} 
\multicolumn{1}{c|}{3} & \multicolumn{1}{c|}{3} & \multicolumn{1}{c|}{4} & \multicolumn{1}{c|}{5} & \multicolumn{1}{c|}{0} & \multicolumn{1}{c|}{1} & \multicolumn{1}{c|}{2} \\ \cline{2-7} 
\multicolumn{1}{c|}{4} & \multicolumn{1}{c|}{4} & \multicolumn{1}{c|}{5} & \multicolumn{1}{c|}{0} & \multicolumn{1}{c|}{1} & \multicolumn{1}{c|}{2} & \multicolumn{1}{c|}{3} \\ \cline{2-7} 
\multicolumn{1}{c|}{5} & \multicolumn{1}{c|}{5} & \multicolumn{1}{c|}{0} & \multicolumn{1}{c|}{1} & \multicolumn{1}{c|}{2} & \multicolumn{1}{c|}{3} & \multicolumn{1}{c|}{4} \\ \cline{2-7} 
\end{tabular}
\end{table}
\end{latin}



\begin{latin}
% Please add the following required packages to your document preamble:
% \usepackage{graphicx}
\begin{table}[H]
\centering
\begin{tabular}{ccccccc}
$\times$               & 0                      & 1                      & 2                      & 3                      & 4                      & 5                      \\ \cline{2-7} 
\multicolumn{1}{c|}{0} & \multicolumn{1}{c|}{0} & \multicolumn{1}{c|}{0} & \multicolumn{1}{c|}{0} & \multicolumn{1}{c|}{0} & \multicolumn{1}{c|}{0} & \multicolumn{1}{c|}{0} \\ \cline{2-7} 
\multicolumn{1}{c|}{1} & \multicolumn{1}{c|}{0} & \multicolumn{1}{c|}{1} & \multicolumn{1}{c|}{2} & \multicolumn{1}{c|}{3} & \multicolumn{1}{c|}{4} & \multicolumn{1}{c|}{5} \\ \cline{2-7} 
\multicolumn{1}{c|}{2} & \multicolumn{1}{c|}{0} & \multicolumn{1}{c|}{2} & \multicolumn{1}{c|}{4} & \multicolumn{1}{c|}{0} & \multicolumn{1}{c|}{2} & \multicolumn{1}{c|}{4} \\ \cline{2-7} 
\multicolumn{1}{c|}{3} & \multicolumn{1}{c|}{0} & \multicolumn{1}{c|}{3} & \multicolumn{1}{c|}{0} & \multicolumn{1}{c|}{3} & \multicolumn{1}{c|}{0} & \multicolumn{1}{c|}{3} \\ \cline{2-7} 
\multicolumn{1}{c|}{4} & \multicolumn{1}{c|}{0} & \multicolumn{1}{c|}{4} & \multicolumn{1}{c|}{2} & \multicolumn{1}{c|}{0} & \multicolumn{1}{c|}{4} & \multicolumn{1}{c|}{2} \\ \cline{2-7} 
\multicolumn{1}{c|}{5} & \multicolumn{1}{c|}{0} & \multicolumn{1}{c|}{5} & \multicolumn{1}{c|}{4} & \multicolumn{1}{c|}{3} & \multicolumn{1}{c|}{2} & \multicolumn{1}{c|}{1} \\ \cline{2-7} 
\end{tabular}
\end{table}
\end{latin}


\subsubsection{سوال 1.7.4}
طبق جدول، 2 در $\mathbb{Z}_4$ و 2، 3 و 4 در $\mathbb{Z}_6$ فاقد وارون ضربی‌اند.\newline
شرط لازم و کافی برای اینکه $a$ به پیمانه‌ی $m$ وارون ضربی داشته باشد این است که این دو عدد نسبت به هم اول باشند. از آنجایی که 5 عدد اول است، همه‌ی اعداد صحیح مثبت کمتر از 5 نسبت به 5 اول‌ند. پس وارون ضربی برای تمامی اعضای غیر صفر در $\mathbb{Z}_5$ موجود است.

\subsection{سوال 1.8}
\begin{latin}
\begin{table}[H]
\centering
\begin{tabular}{|c|c|}
\hline
$\times$ & 5  \\ \hline
0        & 0  \\ \hline
1        & 5  \\ \hline
2        & 10 \\ \hline
3        & 4  \\ \hline
4        & 9  \\ \hline
5        & 3  \\ \hline
6        & 8  \\ \hline
7        & 2  \\ \hline
8        & 7  \\ \hline
9        & 1  \\ \hline
10       & 6  \\ \hline
\end{tabular}
\end{table}

% Please add the following required packages to your document preamble:
% \usepackage{graphicx}
\begin{table}[H]
\centering
\begin{tabular}{|c|c|}
\hline
$\times$ & 5  \\ \hline
0        & 0  \\ \hline
1        & 5  \\ \hline
2        & 10 \\ \hline
3        & 3  \\ \hline
4        & 8  \\ \hline
5        & 1  \\ \hline
6        & 6  \\ \hline
7        & 11 \\ \hline
8        & 4  \\ \hline
9        & 9  \\ \hline
10       & 2  \\ \hline
11       & 7  \\ \hline
\end{tabular}
\end{table}

\begin{table}[H]
\centering
\begin{tabular}{|c|c|}
\hline
$\times$ & 5  \\ \hline
0        & 0  \\ \hline
1        & 5  \\ \hline
2        & 10 \\ \hline
3        & 2  \\ \hline
4        & 7  \\ \hline
5        & 12 \\ \hline
6        & 4  \\ \hline
7        & 9  \\ \hline
8        & 1  \\ \hline
9        & 6  \\ \hline
10       & 11 \\ \hline
11       & 3  \\ \hline
12       & 8  \\ \hline
\end{tabular}
\end{table}
\end{latin}
وارون ضربی 5 در  $\mathbb{Z}_{11}$،  $\mathbb{Z}_{12}$ و  $\mathbb{Z}_{13}$ به ترتیب 9، 5 و 8 است.

\subsection{سوال 1.9}
\subsubsection{سوال 1.9.1}
$3 ^ 2 \equiv 9 \quad mod\:13 \Rightarrow x = 9 \\$

\subsubsection{سوال 1.9.2}
$7 ^ 2 \equiv 10 \quad mod\:13 \Rightarrow x = 10 \\$

\subsubsection{سوال 1.9.3}
$3 ^ {10} \equiv (3 ^ {3}) ^ 3 \times 3 \equiv (27) ^ 3 \times 3 \equiv (1) ^ 3 \times 3 \equiv 3 \quad mod\:13 \Rightarrow x = 3 \\$

\subsubsection{سوال 1.9.4}
$7 ^ {100} \equiv (7 ^ {2}) ^ {50} \equiv (-3) ^ {50} \equiv (3) ^ {50} \equiv (3 ^ {10}) ^ 5 \equiv 3 ^ 5 \equiv  3 ^ 3 \times 3 ^ 2 \equiv 9 \quad mod\:13 \Rightarrow x = 9 \\$

\subsubsection{سوال 1.9.5}
\begin{latin}
\begin{table}[H]
\centering
\begin{tabular}{|c|c|c|c|c|c|}
\hline
$power$ & 1 & 2  & 3 & 4 & 5  \\ \hline
7   & 7 & 10 & 5 & 9 & 11 \\ \hline
\end{tabular}
\end{table}
\end{latin}
$\Rightarrow x = 5$

\subsection{سوال 1.10}

\subsubsection{$m = 4$}
$
\left( 4, 1 \right) = 1 \\
\left( 4, 3 \right) = 1 \\
$


\subsubsection{$m = 5$}
$
\left( 5, 1 \right) = 1 \\
\left( 5, 2 \right) = 1 \\
\left( 5, 3 \right) = 1 \\
\left( 5, 4 \right) = 1 \\
$


\subsubsection{$m = 9$}
$
\left( 9, 1 \right) = 1 \\
\left( 9, 2 \right) = 1 \\
\left( 9, 4 \right) = 1 \\
\left( 9, 5 \right) = 1 \\
\left( 9, 7 \right) = 1 \\
\left( 9, 8 \right) = 1 \\
$

\subsubsection{$m = 26$}
$
\left( 26, 1 \right) = 1 \\
\left( 26, 3 \right) = 1 \\
\left( 26, 5 \right) = 1 \\
\left( 26, 7 \right) = 1 \\
\left( 26, 9 \right) = 1 \\
\left( 26, 11 \right) = 1 \\
\left( 26, 15 \right) = 1 \\
\left( 26, 17 \right) = 1 \\
\left( 26, 19 \right) = 1 \\
\left( 26, 21 \right) = 1 \\
\left( 26, 23 \right) = 1 \\
\left( 26, 25 \right) = 1 \\ \\ \\ \\
$

\subsubsection{\lr{Euler’s phi function}}
$
\phi\left( 4 \right) = 4\prod_{p | 4} \left( 1 - \frac{1}{p} \right) = 2 \\
\phi\left( 5 \right) = 5\prod_{p | 5} \left( 1 - \frac{1}{p} \right) = 4 \\
\phi\left( 9 \right) = 9\prod_{p | 9} \left( 1 - \frac{1}{p} \right) = 6 \\
\phi\left( 26 \right) = 26\prod_{p | 26} \left( 1 - \frac{1}{p} \right) = 12 \\
$



\subsection{سوال 1.13}
$
\left( x_1, y_1 \right) \\
\left( x_2, y_2 \right) \\ \\
y_1 = e_k\left( x_1 \right) \equiv  a x_1 + b \quad mod \: m \\
y_2 = e_k\left( x_2 \right) \equiv  a x_2 + b \quad mod \: m \\ \\
\Rightarrow y_1 - y_2 \equiv a\left( x_1 - x_2 \right) \quad mod \: m \\
\Rightarrow \left( y_1 - y_2 \right) \left( x_1 - x_2 \right) ^ {-1} \equiv a \quad mod \: m \\
$
برای اینکه $a$ وجود داشته باشد، باید $\left( x_1 - x_2 \right)$ وارون داشته باشد. از آنجایی که شرط لازم و کافی برای اینکه $\left( x_1 - x_2 \right)$ به پیمانه‌ی $m$ وارون ضربی داشته باشد این است که این دو عدد نسبت به هم اول باشند. پس \lr{Oscar} با فرض دانستن $m$ باید $x_1$ و $x_2$ را طوری انتخاب کند که داشته باشیم:\\
$
\left( \left( x_1 - x_2 \right), m \right) = 1
$




\section{\lr{CrypTool}}%2
\subsection{}
\subsubsection{\lr{a}}
کلید \lr{Caesar cipher} برابر \lr{M} است که حرف 13ام الفبای انگلیسی است. پس در واقع هر حرفِ الفبا به صورت حلقوی 12 واحد شیفت می‌خورد. پس در نهایت به صورت زیر رمز می‌شود.
\begin{latin}
% Please add the following required packages to your document preamble:
% \usepackage{graphicx}
\begin{table}[H]
\centering
\resizebox{\columnwidth}{!}{%
\begin{tabular}{|c|c|c|c|c|c|c|c|c|c|c|c|c|c|c|c|c|c|c|c|c|}
\hline
x           & A & l & i & r & e & z & a &  & A & b & r & e & h & f & o & r & o & u & s & h \\ \hline
$E_{12}(x)$ & M & x & u & d & q & l & m &  & M & n & d & q & t & r & a & d & a & g & e & t \\ \hline
\end{tabular}%
}
\end{table}
\end{latin}
در نرم افزار \lr{CrypTool} به صورت زیر رمز می‌کنیم.
\begin{figure}[H]
    \centering
    \includegraphics[width=0.75\textwidth]{figures/1a.jpg}
    \caption
	{}
    \label{fig:fig1}
\end{figure}

\begin{figure}[H]
    \centering
    \includegraphics[width=0.75\textwidth]{figures/1b.jpg}
    \caption
	{}
    \label{fig:fig1}
\end{figure}

\begin{figure}[H]
    \centering
    \includegraphics[width=0.75\textwidth]{figures/1c.jpg}
    \caption
	{}
    \label{fig:fig1}
\end{figure}

\subsection{}%2
\begin{latin}
$
9816603 \equiv 17 \:\:\:\:mod\:\: 26
$
\end{latin}
کلید \lr{Substitution cipher} برابر \lr{fharjolyinectzspdbkwxgumvq} و \lr{offset} آن برابر 17 است. در واقع الفبای اگلیسی به ترتیب به \lr{NECTZSPDBKWXGUMVQFHARJOLYI} \lr{map} می‌شود. پس در نهایت به صورت زیر رمز می‌شود.
\begin{latin}
% Please add the following required packages to your document preamble:
% \usepackage{graphicx}
\begin{table}[H]
\centering
\resizebox{\columnwidth}{!}{%
\begin{tabular}{|c|c|c|c|c|c|c|c|c|c|c|c|c|c|c|c|c|c|c|c|c|c|c|c|c|c|c|c|c|c|c|c|c|c|c|c|c|c|c|c|c|c|c|c|c|c|c|c|c|c|c|c|c|c|c|c|c|c|c|c|c|c|c|c|c|c|c|c|c|c|}
\hline
x           & S & u & c & c & e & s & s &  & u & s & u & a & l & l & y &  & c & o & m & e & s &  & t & o &  & t & h & o & s & e &  & w & h & o &  & a & r & e &  & t & o & o &  & b & u & s & y &  & t & o &  & b & e &  & l & o & o & k & i & n & g &  & f & o & r &  & i & t & . \\ \hline
$E(x)$ & H & r & c & c & z & h & h &  & r & h & r & n & x & x & y &  & c & m & g & z & h &  & a & m &  & a & d & m & h & z &  & o & d & m &  & n & f & z &  & a & m & m &  & e & r & h & y &  & a & m &  & e & z &  & x & m & m & w & b & u & p &  & s & m & f &  & b & a & . \\ \hline
\end{tabular}%
}
\end{table}
\end{latin}
در نرم افزار \lr{CrypTool} به صورت زیر رمز می‌کنیم.
\begin{figure}[H]
    \centering
    \includegraphics[width=0.75\textwidth]{figures/2a.jpg}
    \caption
	{}
    \label{fig:fig1}
\end{figure}

\begin{figure}[H]
    \centering
    \includegraphics[width=0.75\textwidth]{figures/2b.jpg}
    \caption
	{}
    \label{fig:fig1}
\end{figure}

\begin{figure}[H]
    \centering
    \includegraphics[width=0.75\textwidth]{figures/2c.jpg}
    \caption
	{}
    \label{fig:fig1}
\end{figure}


\subsection{}%3
\subsubsection{\lr{a}}
در \lr{Vigenère cipher} در الفبای انگلیسی از یک جدول با ابعاد $26 \times 26$ استفاده می‌شود که در سطر $i$ام آن حروف انگلیسی به ترتیب به صورت حلقوی با شروع از حرف $i$ام الفبا نوشته شده است.
\begin{latin}
% Please add the following required packages to your document preamble:
% \usepackage{graphicx}
\begin{table}[H]
\centering
\resizebox{\columnwidth}{!}{%
\begin{tabular}{ccccccccccccccccccccccccccc}
                                & \textbf{A}             & \textbf{B}             & \textbf{C}             & \textbf{D}             & \textbf{E}             & \textbf{F}             & \textbf{G}             & \textbf{H}             & \textbf{I}             & \textbf{J}             & \textbf{K}             & \textbf{L}             & \textbf{M}             & \textbf{N}             & \textbf{O}             & \textbf{P}             & \textbf{Q}             & \textbf{R}             & \textbf{S}             & \textbf{T}             & \textbf{U}             & \textbf{V}             & \textbf{W}             & \textbf{X}             & \textbf{Y}             & \textbf{Z}             \\ \cline{2-27} 
\multicolumn{1}{c|}{\textbf{A}} & \multicolumn{1}{c|}{A} & \multicolumn{1}{c|}{B} & \multicolumn{1}{c|}{C} & \multicolumn{1}{c|}{D} & \multicolumn{1}{c|}{E} & \multicolumn{1}{c|}{F} & \multicolumn{1}{c|}{G} & \multicolumn{1}{c|}{H} & \multicolumn{1}{c|}{I} & \multicolumn{1}{c|}{J} & \multicolumn{1}{c|}{K} & \multicolumn{1}{c|}{L} & \multicolumn{1}{c|}{M} & \multicolumn{1}{c|}{N} & \multicolumn{1}{c|}{O} & \multicolumn{1}{c|}{P} & \multicolumn{1}{c|}{Q} & \multicolumn{1}{c|}{R} & \multicolumn{1}{c|}{S} & \multicolumn{1}{c|}{T} & \multicolumn{1}{c|}{U} & \multicolumn{1}{c|}{V} & \multicolumn{1}{c|}{W} & \multicolumn{1}{c|}{X} & \multicolumn{1}{c|}{Y} & \multicolumn{1}{c|}{Z} \\ \cline{2-27} 
\multicolumn{1}{c|}{\textbf{B}} & \multicolumn{1}{c|}{B} & \multicolumn{1}{c|}{C} & \multicolumn{1}{c|}{D} & \multicolumn{1}{c|}{E} & \multicolumn{1}{c|}{F} & \multicolumn{1}{c|}{G} & \multicolumn{1}{c|}{H} & \multicolumn{1}{c|}{I} & \multicolumn{1}{c|}{J} & \multicolumn{1}{c|}{K} & \multicolumn{1}{c|}{L} & \multicolumn{1}{c|}{M} & \multicolumn{1}{c|}{N} & \multicolumn{1}{c|}{O} & \multicolumn{1}{c|}{P} & \multicolumn{1}{c|}{Q} & \multicolumn{1}{c|}{R} & \multicolumn{1}{c|}{S} & \multicolumn{1}{c|}{T} & \multicolumn{1}{c|}{U} & \multicolumn{1}{c|}{V} & \multicolumn{1}{c|}{W} & \multicolumn{1}{c|}{X} & \multicolumn{1}{c|}{Y} & \multicolumn{1}{c|}{Z} & \multicolumn{1}{c|}{A} \\ \cline{2-27} 
\multicolumn{1}{c|}{\textbf{C}} & \multicolumn{1}{c|}{C} & \multicolumn{1}{c|}{D} & \multicolumn{1}{c|}{E} & \multicolumn{1}{c|}{F} & \multicolumn{1}{c|}{G} & \multicolumn{1}{c|}{H} & \multicolumn{1}{c|}{I} & \multicolumn{1}{c|}{J} & \multicolumn{1}{c|}{K} & \multicolumn{1}{c|}{L} & \multicolumn{1}{c|}{M} & \multicolumn{1}{c|}{N} & \multicolumn{1}{c|}{O} & \multicolumn{1}{c|}{P} & \multicolumn{1}{c|}{Q} & \multicolumn{1}{c|}{R} & \multicolumn{1}{c|}{S} & \multicolumn{1}{c|}{T} & \multicolumn{1}{c|}{U} & \multicolumn{1}{c|}{V} & \multicolumn{1}{c|}{W} & \multicolumn{1}{c|}{X} & \multicolumn{1}{c|}{Y} & \multicolumn{1}{c|}{Z} & \multicolumn{1}{c|}{A} & \multicolumn{1}{c|}{B} \\ \cline{2-27} 
\multicolumn{1}{c|}{\textbf{D}} & \multicolumn{1}{c|}{D} & \multicolumn{1}{c|}{E} & \multicolumn{1}{c|}{F} & \multicolumn{1}{c|}{G} & \multicolumn{1}{c|}{H} & \multicolumn{1}{c|}{I} & \multicolumn{1}{c|}{J} & \multicolumn{1}{c|}{K} & \multicolumn{1}{c|}{L} & \multicolumn{1}{c|}{M} & \multicolumn{1}{c|}{N} & \multicolumn{1}{c|}{O} & \multicolumn{1}{c|}{P} & \multicolumn{1}{c|}{Q} & \multicolumn{1}{c|}{R} & \multicolumn{1}{c|}{S} & \multicolumn{1}{c|}{T} & \multicolumn{1}{c|}{U} & \multicolumn{1}{c|}{V} & \multicolumn{1}{c|}{W} & \multicolumn{1}{c|}{X} & \multicolumn{1}{c|}{Y} & \multicolumn{1}{c|}{Z} & \multicolumn{1}{c|}{A} & \multicolumn{1}{c|}{B} & \multicolumn{1}{c|}{C} \\ \cline{2-27} 
\multicolumn{1}{c|}{\textbf{E}} & \multicolumn{1}{c|}{E} & \multicolumn{1}{c|}{F} & \multicolumn{1}{c|}{G} & \multicolumn{1}{c|}{H} & \multicolumn{1}{c|}{I} & \multicolumn{1}{c|}{J} & \multicolumn{1}{c|}{K} & \multicolumn{1}{c|}{L} & \multicolumn{1}{c|}{M} & \multicolumn{1}{c|}{N} & \multicolumn{1}{c|}{O} & \multicolumn{1}{c|}{P} & \multicolumn{1}{c|}{Q} & \multicolumn{1}{c|}{R} & \multicolumn{1}{c|}{S} & \multicolumn{1}{c|}{T} & \multicolumn{1}{c|}{U} & \multicolumn{1}{c|}{V} & \multicolumn{1}{c|}{W} & \multicolumn{1}{c|}{X} & \multicolumn{1}{c|}{Y} & \multicolumn{1}{c|}{Z} & \multicolumn{1}{c|}{A} & \multicolumn{1}{c|}{B} & \multicolumn{1}{c|}{C} & \multicolumn{1}{c|}{D} \\ \cline{2-27} 
\multicolumn{1}{c|}{\textbf{F}} & \multicolumn{1}{c|}{F} & \multicolumn{1}{c|}{G} & \multicolumn{1}{c|}{H} & \multicolumn{1}{c|}{I} & \multicolumn{1}{c|}{J} & \multicolumn{1}{c|}{K} & \multicolumn{1}{c|}{L} & \multicolumn{1}{c|}{M} & \multicolumn{1}{c|}{N} & \multicolumn{1}{c|}{O} & \multicolumn{1}{c|}{P} & \multicolumn{1}{c|}{Q} & \multicolumn{1}{c|}{R} & \multicolumn{1}{c|}{S} & \multicolumn{1}{c|}{T} & \multicolumn{1}{c|}{U} & \multicolumn{1}{c|}{V} & \multicolumn{1}{c|}{W} & \multicolumn{1}{c|}{X} & \multicolumn{1}{c|}{Y} & \multicolumn{1}{c|}{Z} & \multicolumn{1}{c|}{A} & \multicolumn{1}{c|}{B} & \multicolumn{1}{c|}{C} & \multicolumn{1}{c|}{D} & \multicolumn{1}{c|}{E} \\ \cline{2-27} 
\multicolumn{1}{c|}{\textbf{G}} & \multicolumn{1}{c|}{G} & \multicolumn{1}{c|}{H} & \multicolumn{1}{c|}{I} & \multicolumn{1}{c|}{J} & \multicolumn{1}{c|}{K} & \multicolumn{1}{c|}{L} & \multicolumn{1}{c|}{M} & \multicolumn{1}{c|}{N} & \multicolumn{1}{c|}{O} & \multicolumn{1}{c|}{P} & \multicolumn{1}{c|}{Q} & \multicolumn{1}{c|}{R} & \multicolumn{1}{c|}{S} & \multicolumn{1}{c|}{T} & \multicolumn{1}{c|}{U} & \multicolumn{1}{c|}{V} & \multicolumn{1}{c|}{W} & \multicolumn{1}{c|}{X} & \multicolumn{1}{c|}{Y} & \multicolumn{1}{c|}{Z} & \multicolumn{1}{c|}{A} & \multicolumn{1}{c|}{B} & \multicolumn{1}{c|}{C} & \multicolumn{1}{c|}{D} & \multicolumn{1}{c|}{E} & \multicolumn{1}{c|}{F} \\ \cline{2-27} 
\multicolumn{1}{c|}{\textbf{H}} & \multicolumn{1}{c|}{H} & \multicolumn{1}{c|}{I} & \multicolumn{1}{c|}{J} & \multicolumn{1}{c|}{K} & \multicolumn{1}{c|}{L} & \multicolumn{1}{c|}{M} & \multicolumn{1}{c|}{N} & \multicolumn{1}{c|}{O} & \multicolumn{1}{c|}{P} & \multicolumn{1}{c|}{Q} & \multicolumn{1}{c|}{R} & \multicolumn{1}{c|}{S} & \multicolumn{1}{c|}{T} & \multicolumn{1}{c|}{U} & \multicolumn{1}{c|}{V} & \multicolumn{1}{c|}{W} & \multicolumn{1}{c|}{X} & \multicolumn{1}{c|}{Y} & \multicolumn{1}{c|}{Z} & \multicolumn{1}{c|}{A} & \multicolumn{1}{c|}{B} & \multicolumn{1}{c|}{C} & \multicolumn{1}{c|}{D} & \multicolumn{1}{c|}{E} & \multicolumn{1}{c|}{F} & \multicolumn{1}{c|}{G} \\ \cline{2-27} 
\multicolumn{1}{c|}{\textbf{I}} & \multicolumn{1}{c|}{I} & \multicolumn{1}{c|}{J} & \multicolumn{1}{c|}{K} & \multicolumn{1}{c|}{L} & \multicolumn{1}{c|}{M} & \multicolumn{1}{c|}{N} & \multicolumn{1}{c|}{O} & \multicolumn{1}{c|}{P} & \multicolumn{1}{c|}{Q} & \multicolumn{1}{c|}{R} & \multicolumn{1}{c|}{S} & \multicolumn{1}{c|}{T} & \multicolumn{1}{c|}{U} & \multicolumn{1}{c|}{V} & \multicolumn{1}{c|}{W} & \multicolumn{1}{c|}{X} & \multicolumn{1}{c|}{Y} & \multicolumn{1}{c|}{Z} & \multicolumn{1}{c|}{A} & \multicolumn{1}{c|}{B} & \multicolumn{1}{c|}{C} & \multicolumn{1}{c|}{D} & \multicolumn{1}{c|}{E} & \multicolumn{1}{c|}{F} & \multicolumn{1}{c|}{G} & \multicolumn{1}{c|}{H} \\ \cline{2-27} 
\multicolumn{1}{c|}{\textbf{J}} & \multicolumn{1}{c|}{J} & \multicolumn{1}{c|}{K} & \multicolumn{1}{c|}{L} & \multicolumn{1}{c|}{M} & \multicolumn{1}{c|}{N} & \multicolumn{1}{c|}{O} & \multicolumn{1}{c|}{P} & \multicolumn{1}{c|}{Q} & \multicolumn{1}{c|}{R} & \multicolumn{1}{c|}{S} & \multicolumn{1}{c|}{T} & \multicolumn{1}{c|}{U} & \multicolumn{1}{c|}{V} & \multicolumn{1}{c|}{W} & \multicolumn{1}{c|}{X} & \multicolumn{1}{c|}{Y} & \multicolumn{1}{c|}{Z} & \multicolumn{1}{c|}{A} & \multicolumn{1}{c|}{B} & \multicolumn{1}{c|}{C} & \multicolumn{1}{c|}{D} & \multicolumn{1}{c|}{E} & \multicolumn{1}{c|}{F} & \multicolumn{1}{c|}{G} & \multicolumn{1}{c|}{H} & \multicolumn{1}{c|}{I} \\ \cline{2-27} 
\multicolumn{1}{c|}{\textbf{K}} & \multicolumn{1}{c|}{K} & \multicolumn{1}{c|}{L} & \multicolumn{1}{c|}{M} & \multicolumn{1}{c|}{N} & \multicolumn{1}{c|}{O} & \multicolumn{1}{c|}{P} & \multicolumn{1}{c|}{Q} & \multicolumn{1}{c|}{R} & \multicolumn{1}{c|}{S} & \multicolumn{1}{c|}{T} & \multicolumn{1}{c|}{U} & \multicolumn{1}{c|}{V} & \multicolumn{1}{c|}{W} & \multicolumn{1}{c|}{X} & \multicolumn{1}{c|}{Y} & \multicolumn{1}{c|}{Z} & \multicolumn{1}{c|}{A} & \multicolumn{1}{c|}{B} & \multicolumn{1}{c|}{C} & \multicolumn{1}{c|}{D} & \multicolumn{1}{c|}{E} & \multicolumn{1}{c|}{F} & \multicolumn{1}{c|}{G} & \multicolumn{1}{c|}{H} & \multicolumn{1}{c|}{I} & \multicolumn{1}{c|}{J} \\ \cline{2-27} 
\multicolumn{1}{c|}{\textbf{L}} & \multicolumn{1}{c|}{L} & \multicolumn{1}{c|}{M} & \multicolumn{1}{c|}{N} & \multicolumn{1}{c|}{O} & \multicolumn{1}{c|}{P} & \multicolumn{1}{c|}{Q} & \multicolumn{1}{c|}{R} & \multicolumn{1}{c|}{S} & \multicolumn{1}{c|}{T} & \multicolumn{1}{c|}{U} & \multicolumn{1}{c|}{V} & \multicolumn{1}{c|}{W} & \multicolumn{1}{c|}{X} & \multicolumn{1}{c|}{Y} & \multicolumn{1}{c|}{Z} & \multicolumn{1}{c|}{A} & \multicolumn{1}{c|}{B} & \multicolumn{1}{c|}{C} & \multicolumn{1}{c|}{D} & \multicolumn{1}{c|}{E} & \multicolumn{1}{c|}{F} & \multicolumn{1}{c|}{G} & \multicolumn{1}{c|}{H} & \multicolumn{1}{c|}{I} & \multicolumn{1}{c|}{J} & \multicolumn{1}{c|}{K} \\ \cline{2-27} 
\multicolumn{1}{c|}{\textbf{M}} & \multicolumn{1}{c|}{M} & \multicolumn{1}{c|}{N} & \multicolumn{1}{c|}{O} & \multicolumn{1}{c|}{P} & \multicolumn{1}{c|}{Q} & \multicolumn{1}{c|}{R} & \multicolumn{1}{c|}{S} & \multicolumn{1}{c|}{T} & \multicolumn{1}{c|}{U} & \multicolumn{1}{c|}{V} & \multicolumn{1}{c|}{W} & \multicolumn{1}{c|}{X} & \multicolumn{1}{c|}{Y} & \multicolumn{1}{c|}{Z} & \multicolumn{1}{c|}{A} & \multicolumn{1}{c|}{B} & \multicolumn{1}{c|}{C} & \multicolumn{1}{c|}{D} & \multicolumn{1}{c|}{E} & \multicolumn{1}{c|}{F} & \multicolumn{1}{c|}{G} & \multicolumn{1}{c|}{H} & \multicolumn{1}{c|}{I} & \multicolumn{1}{c|}{J} & \multicolumn{1}{c|}{K} & \multicolumn{1}{c|}{L} \\ \cline{2-27} 
\multicolumn{1}{c|}{\textbf{N}} & \multicolumn{1}{c|}{N} & \multicolumn{1}{c|}{O} & \multicolumn{1}{c|}{P} & \multicolumn{1}{c|}{Q} & \multicolumn{1}{c|}{R} & \multicolumn{1}{c|}{S} & \multicolumn{1}{c|}{T} & \multicolumn{1}{c|}{U} & \multicolumn{1}{c|}{V} & \multicolumn{1}{c|}{W} & \multicolumn{1}{c|}{X} & \multicolumn{1}{c|}{Y} & \multicolumn{1}{c|}{Z} & \multicolumn{1}{c|}{A} & \multicolumn{1}{c|}{B} & \multicolumn{1}{c|}{C} & \multicolumn{1}{c|}{D} & \multicolumn{1}{c|}{E} & \multicolumn{1}{c|}{F} & \multicolumn{1}{c|}{G} & \multicolumn{1}{c|}{H} & \multicolumn{1}{c|}{I} & \multicolumn{1}{c|}{J} & \multicolumn{1}{c|}{K} & \multicolumn{1}{c|}{L} & \multicolumn{1}{c|}{M} \\ \cline{2-27} 
\multicolumn{1}{c|}{\textbf{O}} & \multicolumn{1}{c|}{O} & \multicolumn{1}{c|}{P} & \multicolumn{1}{c|}{Q} & \multicolumn{1}{c|}{R} & \multicolumn{1}{c|}{S} & \multicolumn{1}{c|}{T} & \multicolumn{1}{c|}{U} & \multicolumn{1}{c|}{V} & \multicolumn{1}{c|}{W} & \multicolumn{1}{c|}{X} & \multicolumn{1}{c|}{Y} & \multicolumn{1}{c|}{Z} & \multicolumn{1}{c|}{A} & \multicolumn{1}{c|}{B} & \multicolumn{1}{c|}{C} & \multicolumn{1}{c|}{D} & \multicolumn{1}{c|}{E} & \multicolumn{1}{c|}{F} & \multicolumn{1}{c|}{G} & \multicolumn{1}{c|}{H} & \multicolumn{1}{c|}{I} & \multicolumn{1}{c|}{J} & \multicolumn{1}{c|}{K} & \multicolumn{1}{c|}{L} & \multicolumn{1}{c|}{M} & \multicolumn{1}{c|}{N} \\ \cline{2-27} 
\multicolumn{1}{c|}{\textbf{P}} & \multicolumn{1}{c|}{P} & \multicolumn{1}{c|}{Q} & \multicolumn{1}{c|}{R} & \multicolumn{1}{c|}{S} & \multicolumn{1}{c|}{T} & \multicolumn{1}{c|}{U} & \multicolumn{1}{c|}{V} & \multicolumn{1}{c|}{W} & \multicolumn{1}{c|}{X} & \multicolumn{1}{c|}{Y} & \multicolumn{1}{c|}{Z} & \multicolumn{1}{c|}{A} & \multicolumn{1}{c|}{B} & \multicolumn{1}{c|}{C} & \multicolumn{1}{c|}{D} & \multicolumn{1}{c|}{E} & \multicolumn{1}{c|}{F} & \multicolumn{1}{c|}{G} & \multicolumn{1}{c|}{H} & \multicolumn{1}{c|}{I} & \multicolumn{1}{c|}{J} & \multicolumn{1}{c|}{K} & \multicolumn{1}{c|}{L} & \multicolumn{1}{c|}{M} & \multicolumn{1}{c|}{N} & \multicolumn{1}{c|}{O} \\ \cline{2-27} 
\multicolumn{1}{c|}{\textbf{Q}} & \multicolumn{1}{c|}{Q} & \multicolumn{1}{c|}{R} & \multicolumn{1}{c|}{S} & \multicolumn{1}{c|}{T} & \multicolumn{1}{c|}{U} & \multicolumn{1}{c|}{V} & \multicolumn{1}{c|}{W} & \multicolumn{1}{c|}{X} & \multicolumn{1}{c|}{Y} & \multicolumn{1}{c|}{Z} & \multicolumn{1}{c|}{A} & \multicolumn{1}{c|}{B} & \multicolumn{1}{c|}{C} & \multicolumn{1}{c|}{D} & \multicolumn{1}{c|}{E} & \multicolumn{1}{c|}{F} & \multicolumn{1}{c|}{G} & \multicolumn{1}{c|}{H} & \multicolumn{1}{c|}{I} & \multicolumn{1}{c|}{J} & \multicolumn{1}{c|}{K} & \multicolumn{1}{c|}{L} & \multicolumn{1}{c|}{M} & \multicolumn{1}{c|}{N} & \multicolumn{1}{c|}{O} & \multicolumn{1}{c|}{P} \\ \cline{2-27} 
\multicolumn{1}{c|}{\textbf{R}} & \multicolumn{1}{c|}{R} & \multicolumn{1}{c|}{S} & \multicolumn{1}{c|}{T} & \multicolumn{1}{c|}{U} & \multicolumn{1}{c|}{V} & \multicolumn{1}{c|}{W} & \multicolumn{1}{c|}{X} & \multicolumn{1}{c|}{Y} & \multicolumn{1}{c|}{Z} & \multicolumn{1}{c|}{A} & \multicolumn{1}{c|}{B} & \multicolumn{1}{c|}{C} & \multicolumn{1}{c|}{D} & \multicolumn{1}{c|}{E} & \multicolumn{1}{c|}{F} & \multicolumn{1}{c|}{G} & \multicolumn{1}{c|}{H} & \multicolumn{1}{c|}{I} & \multicolumn{1}{c|}{J} & \multicolumn{1}{c|}{K} & \multicolumn{1}{c|}{L} & \multicolumn{1}{c|}{M} & \multicolumn{1}{c|}{N} & \multicolumn{1}{c|}{O} & \multicolumn{1}{c|}{P} & \multicolumn{1}{c|}{Q} \\ \cline{2-27} 
\multicolumn{1}{c|}{\textbf{S}} & \multicolumn{1}{c|}{S} & \multicolumn{1}{c|}{T} & \multicolumn{1}{c|}{U} & \multicolumn{1}{c|}{V} & \multicolumn{1}{c|}{W} & \multicolumn{1}{c|}{X} & \multicolumn{1}{c|}{Y} & \multicolumn{1}{c|}{Z} & \multicolumn{1}{c|}{A} & \multicolumn{1}{c|}{B} & \multicolumn{1}{c|}{C} & \multicolumn{1}{c|}{D} & \multicolumn{1}{c|}{E} & \multicolumn{1}{c|}{F} & \multicolumn{1}{c|}{G} & \multicolumn{1}{c|}{H} & \multicolumn{1}{c|}{I} & \multicolumn{1}{c|}{J} & \multicolumn{1}{c|}{K} & \multicolumn{1}{c|}{L} & \multicolumn{1}{c|}{M} & \multicolumn{1}{c|}{N} & \multicolumn{1}{c|}{O} & \multicolumn{1}{c|}{P} & \multicolumn{1}{c|}{Q} & \multicolumn{1}{c|}{R} \\ \cline{2-27} 
\multicolumn{1}{c|}{\textbf{T}} & \multicolumn{1}{c|}{T} & \multicolumn{1}{c|}{U} & \multicolumn{1}{c|}{V} & \multicolumn{1}{c|}{W} & \multicolumn{1}{c|}{X} & \multicolumn{1}{c|}{Y} & \multicolumn{1}{c|}{Z} & \multicolumn{1}{c|}{A} & \multicolumn{1}{c|}{B} & \multicolumn{1}{c|}{C} & \multicolumn{1}{c|}{D} & \multicolumn{1}{c|}{E} & \multicolumn{1}{c|}{F} & \multicolumn{1}{c|}{G} & \multicolumn{1}{c|}{H} & \multicolumn{1}{c|}{I} & \multicolumn{1}{c|}{J} & \multicolumn{1}{c|}{K} & \multicolumn{1}{c|}{L} & \multicolumn{1}{c|}{M} & \multicolumn{1}{c|}{N} & \multicolumn{1}{c|}{O} & \multicolumn{1}{c|}{P} & \multicolumn{1}{c|}{Q} & \multicolumn{1}{c|}{R} & \multicolumn{1}{c|}{S} \\ \cline{2-27} 
\multicolumn{1}{c|}{\textbf{U}} & \multicolumn{1}{c|}{U} & \multicolumn{1}{c|}{V} & \multicolumn{1}{c|}{W} & \multicolumn{1}{c|}{X} & \multicolumn{1}{c|}{Y} & \multicolumn{1}{c|}{Z} & \multicolumn{1}{c|}{A} & \multicolumn{1}{c|}{B} & \multicolumn{1}{c|}{C} & \multicolumn{1}{c|}{D} & \multicolumn{1}{c|}{E} & \multicolumn{1}{c|}{F} & \multicolumn{1}{c|}{G} & \multicolumn{1}{c|}{H} & \multicolumn{1}{c|}{I} & \multicolumn{1}{c|}{J} & \multicolumn{1}{c|}{K} & \multicolumn{1}{c|}{L} & \multicolumn{1}{c|}{M} & \multicolumn{1}{c|}{N} & \multicolumn{1}{c|}{O} & \multicolumn{1}{c|}{P} & \multicolumn{1}{c|}{Q} & \multicolumn{1}{c|}{R} & \multicolumn{1}{c|}{S} & \multicolumn{1}{c|}{T} \\ \cline{2-27} 
\multicolumn{1}{c|}{\textbf{V}} & \multicolumn{1}{c|}{V} & \multicolumn{1}{c|}{W} & \multicolumn{1}{c|}{X} & \multicolumn{1}{c|}{Y} & \multicolumn{1}{c|}{Z} & \multicolumn{1}{c|}{A} & \multicolumn{1}{c|}{B} & \multicolumn{1}{c|}{C} & \multicolumn{1}{c|}{D} & \multicolumn{1}{c|}{E} & \multicolumn{1}{c|}{F} & \multicolumn{1}{c|}{G} & \multicolumn{1}{c|}{H} & \multicolumn{1}{c|}{I} & \multicolumn{1}{c|}{J} & \multicolumn{1}{c|}{K} & \multicolumn{1}{c|}{L} & \multicolumn{1}{c|}{M} & \multicolumn{1}{c|}{N} & \multicolumn{1}{c|}{O} & \multicolumn{1}{c|}{P} & \multicolumn{1}{c|}{Q} & \multicolumn{1}{c|}{R} & \multicolumn{1}{c|}{S} & \multicolumn{1}{c|}{T} & \multicolumn{1}{c|}{U} \\ \cline{2-27} 
\multicolumn{1}{c|}{\textbf{W}} & \multicolumn{1}{c|}{W} & \multicolumn{1}{c|}{X} & \multicolumn{1}{c|}{Y} & \multicolumn{1}{c|}{Z} & \multicolumn{1}{c|}{A} & \multicolumn{1}{c|}{B} & \multicolumn{1}{c|}{C} & \multicolumn{1}{c|}{D} & \multicolumn{1}{c|}{E} & \multicolumn{1}{c|}{F} & \multicolumn{1}{c|}{G} & \multicolumn{1}{c|}{H} & \multicolumn{1}{c|}{I} & \multicolumn{1}{c|}{J} & \multicolumn{1}{c|}{K} & \multicolumn{1}{c|}{L} & \multicolumn{1}{c|}{M} & \multicolumn{1}{c|}{N} & \multicolumn{1}{c|}{O} & \multicolumn{1}{c|}{P} & \multicolumn{1}{c|}{Q} & \multicolumn{1}{c|}{R} & \multicolumn{1}{c|}{S} & \multicolumn{1}{c|}{T} & \multicolumn{1}{c|}{U} & \multicolumn{1}{c|}{V} \\ \cline{2-27} 
\multicolumn{1}{c|}{\textbf{X}} & \multicolumn{1}{c|}{X} & \multicolumn{1}{c|}{Y} & \multicolumn{1}{c|}{Z} & \multicolumn{1}{c|}{A} & \multicolumn{1}{c|}{B} & \multicolumn{1}{c|}{C} & \multicolumn{1}{c|}{D} & \multicolumn{1}{c|}{E} & \multicolumn{1}{c|}{F} & \multicolumn{1}{c|}{G} & \multicolumn{1}{c|}{H} & \multicolumn{1}{c|}{I} & \multicolumn{1}{c|}{J} & \multicolumn{1}{c|}{K} & \multicolumn{1}{c|}{L} & \multicolumn{1}{c|}{M} & \multicolumn{1}{c|}{N} & \multicolumn{1}{c|}{O} & \multicolumn{1}{c|}{P} & \multicolumn{1}{c|}{Q} & \multicolumn{1}{c|}{R} & \multicolumn{1}{c|}{S} & \multicolumn{1}{c|}{T} & \multicolumn{1}{c|}{U} & \multicolumn{1}{c|}{V} & \multicolumn{1}{c|}{W} \\ \cline{2-27} 
\multicolumn{1}{c|}{\textbf{Y}} & \multicolumn{1}{c|}{Y} & \multicolumn{1}{c|}{Z} & \multicolumn{1}{c|}{A} & \multicolumn{1}{c|}{B} & \multicolumn{1}{c|}{C} & \multicolumn{1}{c|}{D} & \multicolumn{1}{c|}{E} & \multicolumn{1}{c|}{F} & \multicolumn{1}{c|}{G} & \multicolumn{1}{c|}{H} & \multicolumn{1}{c|}{I} & \multicolumn{1}{c|}{J} & \multicolumn{1}{c|}{K} & \multicolumn{1}{c|}{L} & \multicolumn{1}{c|}{M} & \multicolumn{1}{c|}{N} & \multicolumn{1}{c|}{O} & \multicolumn{1}{c|}{P} & \multicolumn{1}{c|}{Q} & \multicolumn{1}{c|}{R} & \multicolumn{1}{c|}{S} & \multicolumn{1}{c|}{T} & \multicolumn{1}{c|}{U} & \multicolumn{1}{c|}{V} & \multicolumn{1}{c|}{W} & \multicolumn{1}{c|}{X} \\ \cline{2-27} 
\multicolumn{1}{c|}{\textbf{Z}} & \multicolumn{1}{c|}{Z} & \multicolumn{1}{c|}{A} & \multicolumn{1}{c|}{B} & \multicolumn{1}{c|}{C} & \multicolumn{1}{c|}{D} & \multicolumn{1}{c|}{E} & \multicolumn{1}{c|}{F} & \multicolumn{1}{c|}{G} & \multicolumn{1}{c|}{H} & \multicolumn{1}{c|}{I} & \multicolumn{1}{c|}{J} & \multicolumn{1}{c|}{K} & \multicolumn{1}{c|}{L} & \multicolumn{1}{c|}{M} & \multicolumn{1}{c|}{N} & \multicolumn{1}{c|}{O} & \multicolumn{1}{c|}{P} & \multicolumn{1}{c|}{Q} & \multicolumn{1}{c|}{R} & \multicolumn{1}{c|}{S} & \multicolumn{1}{c|}{T} & \multicolumn{1}{c|}{U} & \multicolumn{1}{c|}{V} & \multicolumn{1}{c|}{W} & \multicolumn{1}{c|}{X} & \multicolumn{1}{c|}{Y} \\ \cline{2-27} 
\end{tabular}%
}
\end{table}
\end{latin}
%%%%%%%%%%%%%%
همچنین کلید مورد استفاده در این الگوریتم به صورت زیر (سه حرف اول نام + سه حرف اول نام خانوادگی) ساخته می‌شود.
\begin{latin}
$
ALIREZA\:\:ABREHFOROUSH \Rightarrow key = ALIABR
$
\end{latin}
حال \lr{key} را مکررا تکرار می‌کنیم تا طول آن برابر طول رشته‌ای که می‌خواهیم آن را رمز کنیم بشود (یا به عبارتی کاراکتر نظیر باقیمانده‌ی $i$ به پیمانه‌ی طول کلید (6) را در کلید به دست آوریم). برای رمز کردن کاراکترِ $i$ام در رشته، کاراکترِ اندیسِ باقیمانده‌ی $i$ به پیمانه‌ی طول کلید (6) در کلید ($key_i$) به همراه خود کاراکترِ $i$ام ($x_i$) به دست می‌آوریم. $cipher_i$ نظیر $x_i$ برابر کاراکتر قرار گرفته در سطرِ $key_i$ و ستون $x_i$ است.
\begin{latin}
% Please add the following required packages to your document preamble:
% \usepackage{graphicx}
\begin{table}[H]
\centering
\resizebox{\columnwidth}{!}{%
\begin{tabular}{|c|c|c|c|c|c|c|c|c|c|c|c|c|c|c|c|c|c|c|c|c|c|c|c|c|c|c|c|c|c|c|c|c|c|c|c|c|c|c|c|c|c|c|c|c|c|c|c|c|c|c|c|c|c|c|c|c|c|c|c|c|c|c|c|c|c|c|c|c|c|}
\hline
key    & A & l & i & a & b & r & a &  & l & i & a & b & r & a & l &  & i & a & b & r & a &  & l & i &  & a & b & r & a & l &  & i & a & b &  & r & a & l &  & i & a & b &  & r & a & l & i &  & a & b &  & r & a &  & l & i & a & b & r & a & l &  & i & a & b &  & r & a & . \\ \hline
x      & S & u & c & c & e & s & s &  & u & s & u & a & l & l & y &  & c & o & m & e & s &  & t & o &  & t & h & o & s & e &  & w & h & o &  & a & r & e &  & t & o & o &  & b & u & s & y &  & t & o &  & b & e &  & l & o & o & k & i & n & g &  & f & o & r &  & i & t & . \\ \hline
$E(x)$ & S & f & k & c & f & j & s &  & f & a & u & b & c & l & j &  & k & o & n & v & s &  & e & w &  & t & i & f & s & p &  & e & h & p &  & r & r & p &  & b & o & p &  & s & u & d & g &  & t & p &  & s & e &  & w & w & o & l & z & n & r &  & n & o & s &  & z & t & . \\ \hline
\end{tabular}%
}
\end{table}
\end{latin}
در نرم افزار \lr{CrypTool} به صورت زیر رمز می‌کنیم.
\begin{figure}[H]
    \centering
    \includegraphics[width=0.75\textwidth]{figures/3aa.jpg}
    \caption
	{}
    \label{fig:fig1}
\end{figure}

\begin{figure}[H]
    \centering
    \includegraphics[width=0.75\textwidth]{figures/3ab.jpg}
    \caption
	{}
    \label{fig:fig1}
\end{figure}

\begin{figure}[H]
    \centering
    \includegraphics[width=0.75\textwidth]{figures/3ac.jpg}
    \caption
	{}
    \label{fig:fig1}
\end{figure}

\subsubsection{\lr{b}}
مشابه قسمت قبل (صرفا تغییر کلید) داریم:
\begin{latin}
$
ALIREZA\:\:ABREHFOROUSH \Rightarrow key = ALIREZAABREHFOROUSH
$
\end{latin}

\begin{latin}
% Please add the following required packages to your document preamble:
% \usepackage{graphicx}
\begin{table}[H]
\centering
\resizebox{\columnwidth}{!}{%
\begin{tabular}{|c|c|c|c|c|c|c|c|c|c|c|c|c|c|c|c|c|c|c|c|c|c|c|c|c|c|c|c|c|c|c|c|c|c|c|c|c|c|c|c|c|c|c|c|c|c|c|c|c|c|c|c|c|c|c|c|c|c|c|c|c|c|c|c|c|c|c|c|c|c|}
\hline
key    & A & l & i & r & e & z & a &  & a & b & r & e & h & f & o &  & r & o & u & s & h &  & a & l &  & i & r & e & z & a &  & a & b & r &  & e & h & f &  & o & r & o &  & u & s & h & a &  & l & i &  & r & e &  & z & a & a & b & r & e & h &  & f & o & r &  & o & u & . \\ \hline
x      & S & u & c & c & e & s & s &  & u & s & u & a & l & l & y &  & c & o & m & e & s &  & t & o &  & t & h & o & s & e &  & w & h & o &  & a & r & e &  & t & o & o &  & b & u & s & y &  & t & o &  & b & e &  & l & o & o & k & i & n & g &  & f & o & r &  & i & t & . \\ \hline
$E(x)$ & S & f & k & t & i & r & s &  & u & t & l & e & s & q & m &  & t & c & g & w & z &  & t & z &  & b & y & s & r & e &  & w & i & f &  & e & y & j &  & h & f & c &  & v & m & z & y &  & e & w &  & s & i &  & k & o & o & l & z & r & n &  & k & c & i &  & w & n & . \\ \hline
\end{tabular}%
}
\end{table}
\end{latin}
در نرم افزار \lr{CrypTool} به صورت زیر رمز می‌کنیم.
\begin{figure}[H]
    \centering
    \includegraphics[width=0.75\textwidth]{figures/3ba.jpg}
    \caption
	{}
    \label{fig:fig1}
\end{figure}

\begin{figure}[H]
    \centering
    \includegraphics[width=0.75\textwidth]{figures/3bb.jpg}
    \caption
	{}
    \label{fig:fig1}
\end{figure}

\begin{figure}[H]
    \centering
    \includegraphics[width=0.75\textwidth]{figures/3bc.jpg}
    \caption
	{}
    \label{fig:fig1}
\end{figure}

\subsubsection{}
؟؟؟؟؟؟؟؟؟؟؟؟؟؟؟؟؟؟؟؟؟؟؟؟؟؟؟؟؟؟؟؟؟؟؟؟؟؟؟؟؟؟




\subsection{}%4
در نرم افزار \lr{CrypTool} به صورت زیر رمزگشایی می‌کنیم. طول کلید (به طور پیشفرض) 5 است و کلید در \lr{Vigenère cipher} برابر \lr{SMILE} به دست می‌آید.
\begin{figure}[H]
    \centering
    \includegraphics[width=0.75\textwidth]{figures/4a.jpg}
    \caption
	{}
    \label{fig:fig1}
\end{figure}

\begin{figure}[H]
    \centering
    \includegraphics[width=0.75\textwidth]{figures/4b.jpg}
    \caption
	{}
    \label{fig:fig1}
\end{figure}

\begin{figure}[H]
    \centering
    \includegraphics[width=0.75\textwidth]{figures/4c.jpg}
    \caption
	{}
    \label{fig:fig1}
\end{figure}

\begin{figure}[H]
    \centering
    \includegraphics[width=0.75\textwidth]{figures/4d.jpg}
    \caption
	{}
    \label{fig:fig1}
\end{figure}

\begin{figure}[H]
    \centering
    \includegraphics[width=0.75\textwidth]{figures/4e.jpg}
    \caption
	{}
    \label{fig:fig1}
\end{figure}
نمودار رسم شده \lr{autocorrelation} را نشان می‌دهد.  \lr{autocorrelation} یک متن را با نسخه‌های مختلف شیفت یافته‌ی آن (به طول یکسان) مقایسه می‌کند. در هر حالت کاراکترهایی که باهم \lr{match} می‌شوند (یکسان‌اند) را تعیین می‌کنیم. در نمودار رسم شده، تعداد کاراکترهای \lr{match}شده بر اساس تعداد واحد شیفت داده شده نمایش داده شده است. توجه شود که فقط حروف الفبای انتخاب شده (انگلیسی یا آلمانی برای مثال) تجزیه و تحلیل می‌شوند. همچنین تعداد جابه‌جایی‌ها به طول متن بستگی دارد (شما می‌توانید متنی متشکل از $n$ کاراکتر را حداکثر $n$ واحد جابجا کنید، سپس آن‌ها به نوعی زیر یکدیگر قرار می‌گیرند). به مثال زیر توجه کنید.
\begin{latin}
% Please add the following required packages to your document preamble:
% \usepackage{graphicx}
% \usepackage[table,xcdraw]{xcolor}
% If you use beamer only pass "xcolor=table" option, i.e. \documentclass[xcolor=table]{beamer}
\begin{table}[H]
\centering
\resizebox{\columnwidth}{!}{%
\begin{tabular}{|c|c|c|c|c|c|c|c|c|c|c|c|c|c|c|c|c|c|c|c|c|c|c|c|c|c|c|c|c|c|c|c|c|c|c|c|c|c|c|c|c|c|c|c|c|c|c|c|c|c|c|c|c|c|c|c|c|c|c|c|c|c|c|c|c|c|c|c|c|c|llllll}
\cline{1-70}
Orginal text & S & u & c & c & e & s & s                                  &                                    & u          & s & u & a & l & l & y &   & c & o & m & e & s &   & t & o &                                    & t & h & o & s & e &   & w                                  & h & o &                                    & a & r & e &   & t & o                                  & o                                  &   & b & u & s & y &   & t & o &   & b                                  & e &            & l & o & o & k & i & n & g &   & f & o & r &  & i & t & . &                      &                      &                      &                      &                      &                      \\ \cline{1-70}
Modified     & S & u & c & c & e & s & \cellcolor[HTML]{FFCCC9}\textbf{s} & \cellcolor[HTML]{FFCCC9}\textbf{u} & \textbf{s} & u & a & l & l & y & c & o & m & e & s & t & o & t & h & o & \cellcolor[HTML]{FFCCC9}\textbf{s} & e & w & h & o & a & r & \cellcolor[HTML]{FFCCC9}\textbf{e} & t & o & \cellcolor[HTML]{FFCCC9}\textbf{o} & b & u & s & y & t & \cellcolor[HTML]{FFCCC9}\textbf{o} & \cellcolor[HTML]{FFCCC9}\textbf{b} & e & l & o & o & k & i & n & g & f & \cellcolor[HTML]{FFCCC9}\textbf{o} & r & \textbf{i} & t & . &   &   &   &   &   &   &   &   &   &  &   &   &   &                      &                      &                      &                      &                      &                      \\ \cline{1-70}
Shifted by 6 &   &   &   &   &   &   & \cellcolor[HTML]{FFCCC9}\textbf{S} & \cellcolor[HTML]{FFCCC9}\textbf{u} & c          & c & e & s & s & u & s & u & a & l & l & y & c & o & m & e & \cellcolor[HTML]{FFCCC9}\textbf{s} & t & o & t & h & o & s & \cellcolor[HTML]{FFCCC9}\textbf{e} & w & h & \cellcolor[HTML]{FFCCC9}\textbf{o} & a & r & e & t & o & \cellcolor[HTML]{FFCCC9}\textbf{o} & \cellcolor[HTML]{FFCCC9}\textbf{b} & u & s & y & t & o & b & e & l & o & \cellcolor[HTML]{FFCCC9}\textbf{o} & k & \textbf{i} & n & g & f & o & r & i & t & . &   &   &   &  &   &   &   & \multicolumn{1}{c}{} & \multicolumn{1}{c}{} & \multicolumn{1}{c}{} & \multicolumn{1}{c}{} & \multicolumn{1}{c}{} & \multicolumn{1}{c}{} \\ \cline{1-70}
\end{tabular}%
}
\end{table}
\end{latin}
در این مثال در شیفت 6 واحد، تعداد کاراکترهای \lr{match}شده برابر 8 است.

\subsection{}%5

\subsubsection{\lr{a}}
\lr{plaintext} مذکور را با \lr{OTP Key} مذکور به شکل زیر با تکنیک \lr{one-time pad} رمز می‌کنیم.
\begin{figure}[H]
    \centering
    \includegraphics[width=0.75\textwidth]{figures/5aa.jpg}
    \caption
	{}
    \label{fig:fig1}
\end{figure}
\begin{figure}[H]
    \centering
    \includegraphics[width=0.75\textwidth]{figures/5ab.jpg}
    \caption
	{}
    \label{fig:fig1}
\end{figure}

\subsubsection{\lr{b}}
\lr{plaintext} مذکور را به شکل زیر با تکنیک \lr{one-time pad} رمز می‌کنیم (از آنجایی که طول کلید \lr{OTP} بایستی بزرگتر مساوی طول رشته‌ای که می‌خواهیم رمز کنیم باشد؛ کلید \lr{OTP} را برابر تکرار رشته‌ی \lr{Alireza Abrehforoush} قرار می‌دهیم).
\begin{figure}[H]
    \centering
    \includegraphics[width=0.75\textwidth]{figures/5ba.jpg}
    \caption
	{}
    \label{fig:fig1}
\end{figure}

\subsubsection{\lr{c}}
به شکل زیر تحلیل برای کشف کلید \lr{OTP} به ترتیب برای قسمت \lr{a} و \lr{b} انجام می‌شود.
\begin{figure}[H]
    \centering
    \includegraphics[width=0.75\textwidth]{figures/5ca.jpg}
    \caption
	{}
    \label{fig:fig1}
\end{figure}
\begin{figure}[H]
    \centering
    \includegraphics[width=0.75\textwidth]{figures/5cb.jpg}
    \caption
	{}
    \label{fig:fig1}
\end{figure}
\begin{figure}[H]
    \centering
    \includegraphics[width=0.75\textwidth]{figures/5cc.jpg}
    \caption
	{}
    \label{fig:fig1}
\end{figure}



\begin{figure}[H]
    \centering
    \includegraphics[width=0.75\textwidth]{figures/5cd.jpg}
    \caption
	{}
    \label{fig:fig1}
\end{figure}
\begin{figure}[H]
    \centering
    \includegraphics[width=0.75\textwidth]{figures/5ce.jpg}
    \caption
	{}
    \label{fig:fig1}
\end{figure}
\begin{figure}[H]
    \centering
    \includegraphics[width=0.75\textwidth]{figures/5cf.jpg}
    \caption
	{}
    \label{fig:fig1}
\end{figure}


%\begin{latin}
%\lstinputlisting{sources/p2.m}
%\end{latin}


%%%%%%%%%%%%%%%%%%%%%%%%%%%%%%%%%%%
%%%%%%%%%%%%%%%%%%%%%%%%%%%%%%%%%%%
%%%%%%%%%%%%%%%%%%%%%%%%%%%%%%%%%%%

%------------------------------------------------------------------------------------------


\subsection*{منابع}
\renewcommand{\subsection}[2]{}%
\begin{thebibliography}{99} % assumes less than 100 references
%چنانچه مرجع فارسی نیز داشته باشید باید دستور فوق را فعال کنید و مراجع فارسی خود را بعد از این دستور وارد کنید


\begin{LTRitems}

\resetlatinfont

\bibitem{b1}
\end{LTRitems}

\end{thebibliography}


\end{document}

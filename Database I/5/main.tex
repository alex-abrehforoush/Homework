\documentclass{article}

\usepackage{graphicx}
\usepackage{rotating}
\usepackage{amsmath}
\usepackage{fancyhdr}
\usepackage{listings}
\usepackage{xcolor}
\usepackage{textcomp}
\usepackage{float}
\usepackage[sorting=none]{biblatex}
\usepackage[margin=1in]{geometry}
\usepackage[font={small,it}]{caption}
\usepackage{placeins}
\usepackage{xepersian}

%\DeclareMathOperator*{\btie}{\bowtie}
\addbibresource{bibliography.bib}
\settextfont[Scale=1.2]{B-NAZANIN.TTF}
\setlatintextfont[Scale=1]{Times New Roman}
\renewcommand{\baselinestretch}{1.5}
\pagestyle{fancy}
\fancyhf{}
\rhead{تکلیف پنجم درس پایگاه داده‌ها 1}
\lhead{\thepage}
\rfoot{علیرضا ابره فروش}
\lfoot{9816603}
\renewcommand{\headrulewidth}{1pt}
\renewcommand{\footrulewidth}{1pt}
%%%%%%%%%%
\lstset
{
    language=[latex]tex,
    basicstyle=\ttfamily,
    commentstyle=\color{black},
    columns=fullflexible,
    keepspaces=true,
    upquote=true,
    showstringspaces=false,
    morestring=[s]\\\%,
    stringstyle=\color{black},
}
%%%%%%%%%%

\begin{document}
\begin{titlepage}
\begin{center}
\includegraphics[width=0.4\textwidth]{IUT Logo.png}\\
        
\LARGE
\textbf{دانشگاه صنعتی اصفهان}\\
\textbf{دانشکده مهندسی برق و کامپیوتر}\\
        
\vfill
        
\huge
\textbf{عنوان: تکلیف اول درس سیستم‌های عامل 1}\\
        
\vfill
        
\LARGE
\textbf{نام و نام خانوادگی: علیرضا ابره فروش}\\
\textbf{شماره دانشجویی: 9816603}\\
\textbf{نیم\,سال تحصیلی: پاییز 1400}\\
\textbf{مدرّس: دکتر محمّدرضا حیدرپور}\\
\textbf{دستیاران آموزشی: مجید فرهادی - دانیال مهرآیین - محمّد نعیمی}\\
\end{center}
\end{titlepage}


%\tableofcontents
\newpage


\section{}%1
مراحل ده‌گانه‌ی طراحی پایگاه داده به شرح زیر است.
\begin{enumerate}
    \item \textbf{\lr{Identify Entities}:}
نقش‌ها، رویدادها، مکان‌ها، اشیاءِ ملموس، یا مفاهیمی که نهایتا کاربر درباره‌ی آن‌ها داده نگه‌داری می‌کند را مشخص کنید.
    \item \textbf{\lr{Find Relationships}:}
وابستگی طبیعی بین هر جفت از \lr{entity}ها را با استفاده از یک ماتریس رابطه پیدا کنید.
	\item \textbf{\lr{Draw Rough ERD}:}
\lr{entity}ها را در مستطیل‌ها و روابط را روی قسمت‌های خطی که \lr{entity}ها را به هم ربط می‌دهند قرار دهید.
	\item \textbf{\lr{Fill in Cardinality}:}
تعداد وقوع یک \lr{entity} را به ازای وقوع یکتای \lr{entity} مربوط به آن تعیین کنید.
	\item \textbf{\lr{Define Primary Keys}:}
\lr{attribute} یا \lr{attribute}هایی که وقوع دقیقا یک رکورد از هر \lr{entity} را مشخص می‌کند را مشخص کنید.
	\item \textbf{\lr{Draw Key-Based ERD}:}
روابطِ \lr{Many-to-Many} را حذف کنید و کلید‌های اصلی و خارجی را در هر \lr{entity} وارد کنید.
	\item \textbf{\lr{Identify Attributes}:}
جزئیات اطلاعاتی(فیلدها) که برای سیستم درحال توسعه الزامی هستند را نام‌گذاری کنید.
	\item \textbf{\lr{Map Attributes}:}
هر \lr{attribute} را با دقیقا یک \lr{entity} که آن را توصیف می‌کند، نظیر کنید.
	\item \textbf{\lr{Draw Fully Attributed ERD}:}
\lr{ERD}ِ گام ششم را با \lr{entity}ها و روابط کشف شده در گام هشتم سازگار کنید.
	\item \textbf{\lr{Check Results}:}
آیا \lr{ERD}ِ نهایی سیستم داده را به دقت مجسم می‌کند؟
\end{enumerate}

\section{}%2
\subsection{\lr{Composite Attribute}}
برای پیاده‌سازی فیزیکی این نوع \lr{attribute}ها می‌توان هر عنصرِ آن‌ها را در هر \lr{entity} همانند \lr{attribute}های عادیِ آن \lr{entity} به \lr{entity} اضافه می‌کرد. در طراحی برای متمایز کردنشان می‌توان از \lr{indent} استفاده کرد.
\subsection{\lr{Multivalued Attribute}}
برای پیاده‌سازی فیزیکی این نوع \lr{attribute}ها می‌توان یک \lr{table} متناظر با آن \lr{attribute} ایجاد کرد و کلید خارجی از \lr{entity}ی اولیه برای آن تعریف کرد.

\section{}%3
\begin{latin}
$
F=
\{
	\{M\}\longrightarrow\{Q\},
	\{Q\}\longrightarrow\{N\},
	\{N\}\longrightarrow\{L, M\},
	\{N\}\longrightarrow\{L\},
	\{P\}\longrightarrow\{L\},
	\{P\}\longrightarrow\{N\},
\}
$
\end{latin}
ابتدا وابستگی‌های تابعی را به گونه‌ای که تنها یک \lr{attribute} در سمت راست آن‌ها قرار داشته باشد بازنویسی می‌کنیم.
\begin{latin}
$
\\F=\\
\{\\
	\{M\}\longrightarrow\{Q\},\\
	\{Q\}\longrightarrow\{N\},\\
	\{N\}\longrightarrow\{L\},\\
	\{N\}\longrightarrow\{M\},\\
	\{N\}\longrightarrow\{L\},\\
	\{P\}\longrightarrow\{L\},\\
	\{P\}\longrightarrow\{N\},\\
\}
$
\end{latin}
حال وابستگی‌های بدیهی را حذف می‌کنیم(چون هیچ وابستگی‌ای که سمت راستش در سمت چپش وجود داشته باشد نداریم پس وابستگی بدیهی وجود ندارد).
\begin{latin}
$
\\F=\\
\{\\
	\{M\}\longrightarrow\{Q\},\\
	\{Q\}\longrightarrow\{N\},\\
	\{N\}\longrightarrow\{L\},\\
	\{N\}\longrightarrow\{M\},\\
	\{N\}\longrightarrow\{L\},\\
	\{P\}\longrightarrow\{L\},\\
	\{P\}\longrightarrow\{N\},\\
\}
$
\end{latin}

سپس سمت چپ هر یک از وابستگی‌ها را کمینه می‌کنیم(کمینه هستند).
\begin{latin}
$
\\F=\\
\{\\
	\{M\}\longrightarrow\{Q\},\\
	\{Q\}\longrightarrow\{N\},\\
	\{N\}\longrightarrow\{L\},\\
	\{N\}\longrightarrow\{M\},\\
	\{N\}\longrightarrow\{L\},\\
	\{P\}\longrightarrow\{L\},\\
	\{P\}\longrightarrow\{N\},\\
\}
$
\end{latin}

در آخر وابستگی‌های تکراری را حذف می‌کنیم.
\begin{latin}
$
\\F_{c}=\\
\{\\
	\{M\}\longrightarrow\{Q\},\\
	\{Q\}\longrightarrow\{N\},\\
	\{N\}\longrightarrow\{M\},\\
	\{N\}\longrightarrow\{L\},\\
	\{P\}\longrightarrow\{N\},\\
\}
$
\end{latin}


\section{}%4
ابتدا پوش کانونی را به شکل زیر محاسبه می‌کنیم:
%\newline
\begin{itemize}
    \item [$\bullet$] ابتدا وابستگی تابعی را به یک وابستگی تابعی با یک \lr{attribute} در سمت راست تبدیل می‌کنیم.

\begin{latin}
$
\\F=\\
\{\\
\{A, B\}\longrightarrow\{C\},\\
\{A\}\longrightarrow\{D\},\\
\{A\}\longrightarrow\{E\},\\
\{B\}\longrightarrow\{F\},\\
\{F\}\longrightarrow\{G\},\\
\{F\}\longrightarrow\{H\},\\
\{D\}\longrightarrow\{I\},\\
\{D\}\longrightarrow\{J\}\\
\}
$
\end{latin}
    \item [$\bullet$] حال وابستگی‌های تابعی بدیهی را پاک می‌کنیم.
\begin{latin}
$
\\F=\\
\{\\
\{A, B\}\longrightarrow\{C\},\\
\{A\}\longrightarrow\{D\},\\
\{A\}\longrightarrow\{E\},\\
\{B\}\longrightarrow\{F\},\\
\{F\}\longrightarrow\{G\},\\
\{F\}\longrightarrow\{H\},\\
\{D\}\longrightarrow\{I\},\\
\{D\}\longrightarrow\{J\}\\
\}
$
\end{latin}
    \item [$\bullet$] سپس \lr{attribute}های سمت چپ هر وابستگی تابعی را کمینه می‌کنیم.

\begin{latin}
$
\\F=\\
\{\\
\{A, B\}\longrightarrow\{C\},\\
\{A\}\longrightarrow\{D\},\\
\{A\}\longrightarrow\{E\},\\
\{B\}\longrightarrow\{F\},\\
\{F\}\longrightarrow\{G\},\\
\{F\}\longrightarrow\{H\},\\
\{D\}\longrightarrow\{I\},\\
\{D\}\longrightarrow\{J\}\\
\}
$
\end{latin}
    \item [$\bullet$] در نهایت وابستگی‌های تابعی تکراری(که از سایر وابستگی‌ها نتیجه می‌شوند) را حذف می‌کنیم.

\begin{latin}
$
\\F_{c}=\\
\{\\
\{A, B\}\longrightarrow\{C\},\\
\{A\}\longrightarrow\{D\},\\
\{A\}\longrightarrow\{E\},\\
\{B\}\longrightarrow\{F\},\\
\{F\}\longrightarrow\{G\},\\
\{F\}\longrightarrow\{H\},\\
\{D\}\longrightarrow\{I\},\\
\{D\}\longrightarrow\{J\}\\
\}
$
\end{latin}
\end{itemize}
پوش کانونی به دست آمد. حال مجموعه‌ی همه‌ی \lr{attribute}هایی که در سمت راست هیچ وابستگی تابعی قرار ندارند را به دست می‌آوریم. هر کلید کاندید باید شامل این \lr{attribute}ها باشد. این مجموعه برابر است با:
\begin{latin}
$
\{A, B\}
$
\end{latin}
$
\{A, B\}
$
سوپرکلید است، پس تنها کلید کاندید است.
\newline
حال با تجزیه‌ی $R$ به \lr{relation}های 
$R_{1}$، $R_{2}$ و $R_{3}$
و $F$ به
 \lr{FD}های $F_{1}$، $F_{2}$ و $F_{3}$ 
به شکل زیر،
 \lr{partial dependency}ها را حذف می‌کنیم و به فرم نرمال دوم می‌رسیم.

\begin{latin}
$
\\
R_{1}=\{A, D, E, I, J\}\\
F_{1}=\\
\{\\
\{A\}\longrightarrow\{D, E\},\\
\{D\}\longrightarrow\{I, J\},\\
\}
\\
\\
R_{2}=\{B, F, G, H\}\\
F_{2}=\\
\{\\
\{B\}\longrightarrow\{F\},\\
\{F\}\longrightarrow\{G, H\},\\
\}
\\
\\
R_{3}=\{A, B, C\}\\
F_{3}=\\
\{\\
\{A, B\}\longrightarrow\{C\},\\
\}
$
\end{latin}
برای دستیابی به فرم نرمال سوم باید \lr{transitive dependency}ها را حذف کنیم. با توجه به اینکه هیچ \lr{FD}ای وجود ندارد که سمت چپ آن یک \lr{attribute}ِ \lr{nonprime} باشد، فرم نرمال سوم همان فرم به دست آمده در مرحله قبل است.
\begin{latin}
$
\\
R_{1}=\{A, D, E, I, J\}\\
F_{1}=\\
\{\\
\{A\}\longrightarrow\{D, E\},\\
\{D\}\longrightarrow\{I, J\},\\
\}
\\
\\
R_{2}=\{B, F, G, H\}\\
F_{2}=\\
\{\\
\{B\}\longrightarrow\{F\},\\
\{F\}\longrightarrow\{G, H\},\\
\}
\\
\\
R_{3}=\{A, B, C\}\\
F_{3}=\\
\{\\
\{A, B\}\longrightarrow\{C\},\\
\}
$
\end{latin}



\section{}%5
\begin{latin}
$
\\
R=
\{
Course\_no, Sec\_no, Offering\_dept, Credit\_hours, Course\_level,\\Instructor\_ssn, Semester, Year, Days\_hours, Room\_no, No\_of\_students
\}\\
\\
F=\\
\{\\
\{Course\_no\}\longrightarrow\{Offering\_dept, Credit\_hours, Course\_level\},\\
\{Course\_no, Sec\_no, Semester, Year\}\longrightarrow\{Days\_hours, Room\_no, No\_of\_students, Instructor\_ssn\},\\
\{Room\_no, Days\_hours, Semester, Year\}\longrightarrow\{Instructor\_ssn, Course\_no, Sec\_no\}\\
\}
$
\end{latin}

ابتدا پوش کانونی را به شکل زیر محاسبه می‌کنیم:

\begin{itemize}
    \item [$\bullet$] ابتدا وابستگی تابعی را به یک وابستگی تابعی با یک \lr{attribute} در سمت راست تبدیل می‌کنیم.

\begin{latin}
$
\\F=\\
\{\\
\{Course\_no\}\longrightarrow\{Offering\_dept\},\\
\{Course\_no \}\longrightarrow\{Credit\_hours\},\\
\{Course\_no \}\longrightarrow\{Course\_level\},\\
\{Course\_no, Sec\_no, Semester, Year \}\longrightarrow\{Days\_hours\},\\
\{Course\_no, Sec\_no, Semester, Year \}\longrightarrow\{Room\_no\},\\
\{Course\_no, Sec\_no, Semester, Year \}\longrightarrow\{No\_of\_students\},\\
\{Course\_no, Sec\_no, Semester, Year \}\longrightarrow\{Instructor\_ssn\},\\
\{Room\_no, Days\_hours, Semester, Year \}\longrightarrow\{Instructor\_ssn\}\\
\{Room\_no, Days\_hours, Semester, Year \}\longrightarrow\{Course\_no\}\\
\{Room\_no, Days\_hours, Semester, Year \}\longrightarrow\{Sec\_no\}\\
\}
$
\end{latin}
    \item [$\bullet$] حال وابستگی‌های تابعی بدیهی را پاک می‌کنیم.
\begin{latin}
$
\\F=\\
\{\\
\{Course\_no\}\longrightarrow\{Offering\_dept\},\\
\{Course\_no \}\longrightarrow\{Credit\_hours\},\\
\{Course\_no \}\longrightarrow\{Course\_level\},\\
\{Course\_no, Sec\_no, Semester, Year \}\longrightarrow\{Days\_hours\},\\
\{Course\_no, Sec\_no, Semester, Year \}\longrightarrow\{Room\_no\},\\
\{Course\_no, Sec\_no, Semester, Year \}\longrightarrow\{No\_of\_students\},\\
\{Course\_no, Sec\_no, Semester, Year \}\longrightarrow\{Instructor\_ssn\},\\
\{Room\_no, Days\_hours, Semester, Year \}\longrightarrow\{Instructor\_ssn\}\\
\{Room\_no, Days\_hours, Semester, Year \}\longrightarrow\{Course\_no\}\\
\{Room\_no, Days\_hours, Semester, Year \}\longrightarrow\{Sec\_no\}\\
\}
$
\end{latin}
    \item [$\bullet$] سپس \lr{attribute}های سمت چپ هر وابستگی تابعی را کمینه می‌کنیم.

\begin{latin}
$
\\F=\\
\{\\
\{Course\_no\}\longrightarrow\{Offering\_dept\},\\
\{Course\_no \}\longrightarrow\{Credit\_hours\},\\
\{Course\_no \}\longrightarrow\{Course\_level\},\\
\{Course\_no, Sec\_no, Semester, Year \}\longrightarrow\{Days\_hours\},\\
\{Course\_no, Sec\_no, Semester, Year \}\longrightarrow\{Room\_no\},\\
\{Course\_no, Sec\_no, Semester, Year \}\longrightarrow\{No\_of\_students\},\\
\{Course\_no, Sec\_no, Semester, Year \}\longrightarrow\{Instructor\_ssn\},\\
\{Room\_no, Days\_hours, Semester, Year \}\longrightarrow\{Instructor\_ssn\}\\
\{Room\_no, Days\_hours, Semester, Year \}\longrightarrow\{Course\_no\}\\
\{Room\_no, Days\_hours, Semester, Year \}\longrightarrow\{Sec\_no\}\\
\}
$
\end{latin}
    \item [$\bullet$] در نهایت وابستگی‌های تابعی تکراری(که از سایر وابستگی‌ها نتیجه می‌شوند) را حذف می‌کنیم.

\begin{latin}
$
\\F_{c}=\\
\{\\
\{Course\_no\}\longrightarrow\{Offering\_dept\},\\
\{Course\_no \}\longrightarrow\{Credit\_hours\},\\
\{Course\_no \}\longrightarrow\{Course\_level\},\\
\{Course\_no, Sec\_no, Semester, Year \}\longrightarrow\{Days\_hours\},\\
\{Course\_no, Sec\_no, Semester, Year \}\longrightarrow\{Room\_no\},\\
\{Course\_no, Sec\_no, Semester, Year \}\longrightarrow\{No\_of\_students\},\\
\{Room\_no, Days\_hours, Semester, Year \}\longrightarrow\{Instructor\_ssn\}\\
\{Room\_no, Days\_hours, Semester, Year \}\longrightarrow\{Course\_no\}\\
\{Room\_no, Days\_hours, Semester, Year \}\longrightarrow\{Sec\_no\}\\
\}
$
\end{latin}
\end{itemize}
پوش کانونی به دست آمد. حال مجموعه‌ی همه‌ی \lr{attribute}هایی که در سمت راست هیچ وابستگی تابعی قرار ندارند را به دست می‌آوریم. هر کلید کاندید باید شامل این \lr{attribute}ها باشد. این مجموعه برابر است با:
\begin{latin}
$
\{Semester, Year\}
$
\end{latin}
همچنین مجموعه‌ی همه‌ی \lr{attribute}هایی که در سمت راست حداقل یک وابستگی تابعی قرار داشته باشد ولی در سمت چپ هیچ وابستگی تابعی قرار ندارند را نیز به دست می‌آوریم. این \lr{attribute}ها نباید در هیچ یک از کلیدهای کاندید باشند. این مجموعه برابر است با:
\begin{latin}
$
\{Offering\_dept, Credit\_hours, Course\_level, No\_of\_students, Instructor\_ssn\}
$
\end{latin}
بستارِ مجموعه‌ی 
$
\{Semester, Year\}
$
خودش است.
حال تلاش می‌کنیم که یکی از \lr{attribute}های مجموعه‌ی
\begin{latin}
$
R-\{Offering\_dept, Credit\_hours, Course\_level, No\_of\_students, Instructor\_ssn\}-\{Semester, Year\}\\
=\{Course\_no, Sec\_no, Days\_hours, Room\_no\}
$
\end{latin}
را به مجموعه‌ی
$
\{Semester, Year\}
$
به گونه‌ای اضافه کنیم که یک سوپرکلید تشکیل دهند. درصورتی که سوپرکلید باشند با بررسی اینکه آیا زیرمجموعه‌ی سره‌ای که سوپرکلید باشند دارند یا خیر کلید کاندید بودن آن‌ها را احراز می‌کنیم.
\begin{itemize}
    \item [$\bullet$]
\begin{latin}
$
\{Semester, Year\}\cup\{Course\_no\}
$
\end{latin}
مجموعه‌ی بالا سوپرکلید نیست. پس کلید کاندید نیست.

\begin{latin}
$
\{Semester, Year\}\cup\{Sec\_no\}
$
\end{latin}
مجموعه‌ی بالا سوپرکلید نیست. پس کلید کاندید نیست.

\begin{latin}
$
\{Semester, Year\}\cup\{Days\_hours\}
$
\end{latin}
مجموعه‌ی بالا سوپرکلید نیست. پس کلید کاندید نیست.

\begin{latin}
$
\{Semester, Year\}\cup\{Room\_no\}
$
\end{latin}
مجموعه‌ی بالا سوپرکلید نیست. پس کلید کاندید نیست.
\end{itemize}
حال دو \lr{attribute} را از مجموعه‌ی مذکور به 
مجموعه‌ی
$
\{Semester, Year\}
$
اضافه می‌کنیم. پس 6 حالت زیر را داریم:
\begin{itemize}
    \item [$\bullet$]
\begin{latin}
$
\{Semester, Year\}\cup\{Course\_no, Sec\_no\}
$
\end{latin}
مجموعه‌ی بالا سوپرکلید است و هیچ زیرمجموعه‌ی سره‌ای که سوپرکلید باشد ندارد. پس یک کلید کاندید است.
\begin{latin}
$
\{Semester, Year\}\cup\{Course\_no, Days\_hours\}
$
\end{latin}
مجموعه‌ی بالا سوپرکلید نیست. پس کلید کاندید نیست.
\begin{latin}
$
\{Semester, Year\}\cup\{Course\_no, Room\_no\}
$
\end{latin}
مجموعه‌ی بالا سوپرکلید نیست. پس کلید کاندید نیست.

\begin{latin}
$
\{Semester, Year\}\cup\{Days\_hours, Sec\_no\}
$
\end{latin}
مجموعه‌ی بالا سوپرکلید نیست. پس کلید کاندید نیست.

\begin{latin}
$
\{Semester, Year\}\cup\{Room\_no, Sec\_no\}
$
\end{latin}
مجموعه‌ی بالا سوپرکلید نیست. پس کلید کاندید نیست.
\begin{latin}
$
\{Semester, Year\}\cup\{Days\_hours, Room\_no\}
$
\end{latin}
مجموعه‌ی بالا سوپرکلید است و هیچ زیرمجموعه‌ی سره‌ای که سوپرکلید باشد ندارد. پس یک کلید کاندید است.
\end{itemize}

حال دو \lr{attribute} را از مجموعه‌ی مذکور به مجموعه‌ی
$
\{Semester, Year\}
$
اضافه می‌کنیم.
\begin{itemize}
    \item [$\bullet$]
\begin{latin}
$
\{Semester, Year\}\cup\{Course\_no, Days\_hours, Room\_no\}
$
\end{latin}
مجموعه‌ی بالا سوپرکلید است اما یک زیرمجموعه‌ی سره دارد که سوپرکلید است. پس کلید کاندید نیست.
\end{itemize}
 چون دیگر کلید کاندید نداریم دیگر ادامه نمی‌دهیم. پس کلیدهای کاندید ما دو مجموعه‌ی زیر هستند:
\begin{latin}
$
\\CK_{1}=\{Semester, Year, Course\_no, Sec\_no\}\\
CK_{2}=\{Semester, Year, Days\_hours, Room\_no\}
$
\end{latin}
%%%%%%%%%%%%%%%%%%%%%%%%%%%%%%%%%
حال با تجزیه‌ی $R$ به \lr{relation}های 
$R_{1}$ و $R_{2}$ 
و $F$ به \lr{FD}های 
$F_{1}$ و $F_{2}$ 
به شکل زیر، \lr{partial dependency}ها را حذف می‌کنیم و به فرم نرمال دوم می‌رسیم.

\begin{latin}
$
\\
R_{1}=\{Course\_no, Offering\_dept, Credit\_hours, Course\_level\}\\
F_{1}=\\
\{\\
\{Course\_no\}\longrightarrow\{Offering\_dept, Credit\_hours, Course\_level\},\\
\}
\\
\\
R_{2}=\{Course\_no, Sec\_no, Instructor\_ssn, Semester, Year, Days\_hours, Room\_no, No\_of\_students\}\\
F_{2}=\\
\{\\
\{Course\_no, Sec\_no, Semester, Year\}\longrightarrow\{Room\_no, Days\_hours\},\\
\{Days\_hours, Room\_no, Semester, Year\}\longrightarrow\{Sec\_no, Course\_no, Instructor\_ssn, No\_of\_students\},\\
\}
$
\end{latin}
برای دستیابی به فرم نرمال سوم باید \lr{transitive dependency}ها را حذف کنیم. با توجه به اینکه هیچ \lr{FD}ای وجود ندارد که سمت چپ آن یک \lr{attribute}ِ \lr{nonprime} باشد، فرم نرمال سوم همان فرم به دست آمده در مرحله قبل است.
\begin{latin}
$
\\
R_{1}=\{Course\_no, Offering\_dept, Credit\_hours, Course\_level\}\\
F_{1}=\\
\{\\
\{Course\_no\}\longrightarrow\{Offering\_dept, Credit\_hours, Course\_level\},\\
\}
\\
\\
R_{2}=\{Course\_no, Sec\_no, Instructor\_ssn, Semester, Year, Days\_hours, Room\_no, No\_of\_students\}\\
F_{2}=\\
\{\\
\{Course\_no, Sec\_no, Semester, Year\}\longrightarrow\{Room\_no, Days\_hours\},\\
\{Days\_hours, Room\_no, Semester, Year\}\longrightarrow\{Sec\_no, Course\_no, Instructor\_ssn, No\_of\_students\},\\
\}
$
\end{latin}



\section{}%6
\subsection{\lr{a}}
\begin{latin}
$
\\
REFRIG=
\{
Model\#, Year, Price, Manuf\_plant, Color
\}
=\{M, Y, P, MP, C\}
\\
\\
F=\\
\{\\
\{M\}\longrightarrow\{MP\},\\
\{M, Y\}\longrightarrow\{P\},\\
\{MP\}\longrightarrow\{C\}\\
\}
$
\end{latin}
ابتدا پوش کانونی را به دست می‌آوریم:
\begin{latin}
$
\\F_{c}=\\
\{\\
\{M\}\longrightarrow\{MP\},\\
\{M, Y\}\longrightarrow\{P\},\\
\{MP\}\longrightarrow\{C\}\\
\}
$
\end{latin}

\begin{itemize}
    \item [$\bullet$] 
\lr{\{M\}} سوپرکلید نیست. چون قادر نیست به صورت یکتا \lr{attribute}ِ \lr{Y} و \lr{P} را تعیین کند.

    \item [$\bullet$] 
\lr{\{M, Y\}} سوپرکلید است. چون قادر است به صورت یکتا همه‌ی \lr{attribute}ها را تعیین کند. همچنین هیچ زیرمجموعه‌ی سره‌ای ندارد که سوپرکلید باشد. پس کلید کاندید است.

    \item [$\bullet$] 
\lr{\{M, C\}} سوپرکلید نیست. چون قادر نیست به صورت یکتا همه‌ی \lr{attribute}ها را تعیین کند(\lr{attribute}ِ \lr{Y} و \lr{P}).
\end{itemize}
\subsection{\lr{b}}
\begin{itemize}
    \item [$\bullet$] 
همه‌ی کلیدهای کاندید را پیدا می‌کنیم. تنها کلید کاندید، کلیدِ
\lr{\{M, Y\}}
است. حال به ازای هر \lr{FD} بررسی می‌کنیم که آیا سمت چپ آن یک سوپرکلید است یا سمت راست آن شامل همه‌ی \lr{attribute}های کلید است یا خیر. چون \lr{FD}ی
$
\{M\}\longrightarrow\{MP\}
$
غیر بدیهی است، سمت چپ آن یک سوپرکلید نیست و سمت راست آن شامل یک \lr{attribute}ِ \lr{nonprime} است، \lr{3NF} نقض می‌شود. پس رابطه‌ی \lr{REFRIG} در \lr{3NF} نیست.
    \item [$\bullet$] 
یک \lr{relation} در \lr{BCNF} است اگر و تنها اگر سمت چپ هر یک از \lr{FD}های غیر بدیهی یک سوپرکلید وجود داشته باشد. چون \lr{FD}ی
$
\{M\}\longrightarrow\{MP\}
$
غیر بدیهی است و سمت چپ آن یک سوپرکلید نیست، این \lr{BCNF} را نقض می‌کند. پس رابطه‌ی \lr{REFRIG} در \lr{BCNF} نیست.
\end{itemize}
%------------------------------------------------------------------------------------------


\section*{منابع}
\renewcommand{\section}[2]{}%
\begin{thebibliography}{99} % assumes less than 100 references
%چنانچه مرجع فارسی نیز داشته باشید باید دستور فوق را فعال کنید و مراجع فارسی خود را بعد از این دستور وارد کنید


\begin{LTRitems}

\resetlatinfont

\bibitem{b1} None
\end{LTRitems}

\end{thebibliography}


\end{document}

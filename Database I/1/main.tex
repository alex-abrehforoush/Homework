\documentclass{article}

\usepackage{graphicx}
\usepackage{fancyhdr}
\usepackage[sorting=none]{biblatex}
\usepackage[margin=1in]{geometry}
\usepackage{xepersian}

\addbibresource{bibliography.bib}
\settextfont[Scale=1.2]{B-NAZANIN.TTF}
\setlatintextfont[Scale=1]{Times New Roman}
\renewcommand{\baselinestretch}{1.5}
\pagestyle{fancy}
\fancyhf{}
\rhead{تکلیف اول درس پایگاه داده‌ها 1}
\lhead{\thepage}
\rfoot{علیرضا ابره فروش}
\lfoot{9816603}
\renewcommand{\headrulewidth}{1pt}
\renewcommand{\footrulewidth}{1pt}

\begin{document}
\begin{titlepage}
\begin{center}
\includegraphics[width=0.4\textwidth]{IUT Logo.png}\\
        
\LARGE
\textbf{دانشگاه صنعتی اصفهان}\\
\textbf{دانشکده مهندسی برق و کامپیوتر}\\
        
\vfill
        
\huge
\textbf{عنوان: تکلیف اول درس سیستم‌های عامل 1}\\
        
\vfill
        
\LARGE
\textbf{نام و نام خانوادگی: علیرضا ابره فروش}\\
\textbf{شماره دانشجویی: 9816603}\\
\textbf{نیم\,سال تحصیلی: پاییز 1400}\\
\textbf{مدرّس: دکتر محمّدرضا حیدرپور}\\
\textbf{دستیاران آموزشی: مجید فرهادی - دانیال مهرآیین - محمّد نعیمی}\\
\end{center}
\end{titlepage}


%\tableofcontents
\newpage

\section{}
اگر سوال بخش\,بندی\,شده نباشد، پاسخ آن در این قسمت نوشته می\,شود.
\subsection{}
پاسخ بخش اول سوال در این قسمت نوشته می\,شود.
\subsection{}
پاسخ بخش دوم سوال در این قسمت نوشته می\,شود.
\subsection{}
پاسخ بخش دوم سوال در این قسمت نوشته می\,شود.

\section{}
در این قسمت با نحوه نوشتن متون دارای کلمات انگلیسی آشنا می\,شوید:\\
\begin{table}[ht]
    \centering
    \begin{tabular}{|c|c|c|}
    \hline
    ویژگی‌ها & معماری دولایه & معماری سه‌لایه\\
    \hline
    سرعت & خانه شماره 5 & خانه شماره 6\\
    \hline
    امنیت & خانه شماره 8 & خانه شماره 9\\
    \hline
    افزونگی & خانه شماره 5 & خانه شماره 6\\
    \hline
    مقیاس‌پذیری & خانه شماره 8 & خانه شماره 9\\
    \hline
    انعطاف‌پذیری & خانه شماره 5 & خانه شماره 6\\
    \hline
    یک‌پارچگی & خانه شماره 8 & خانه شماره 9\\
    \hline
    \end{tabular}
    \caption{جدول شماره 1}
    \label{tab:tab1}
\end{table}

\section{}
در این قسمت با نحوه درج فرمول\,های ریاضی آشنا می\,شوید:
\begin{center}
$E = m{c}^{2}$
\end{center}

\section{}
در این قسمت با نحوه درج اشکال آشنا می\,شوید:
\begin{figure}[ht]
    \centering
    \includegraphics[width=0.4\textwidth]{IUT Logo.png}
    \caption{شکل شماره 1}
    \label{fig:fig1}
\end{figure}

\section{}
در این قسمت با نحوه درج جداول آشنا می\,شوید:
\begin{table}[ht]
    \centering
    \begin{tabular}{|c|c|c|}
    \hline
    خانه شماره 1 & خانه شماره 2 & خانه شماره 3\\
    \hline
    خانه شماره 4 & خانه شماره 5 & خانه شماره 6\\
    \hline
    خانه شماره 7 & خانه شماره 8 & خانه شماره 9\\
    \hline
    \end{tabular}
    \caption{جدول شماره 1}
    \label{tab:tab1}
\end{table}

\section{}
در این قسمت با نحوه درج انواع لیست\,ها آشنا می\,شوید:
\subsection{}
\begin{itemize}
    \item [$\bullet$] مورد اول
    \item [$\bullet$] مورد دوم
\end{itemize}
\subsection{}
\begin{enumerate}
    \item مورد شماره 1
    \item مورد شماره 2
\end{enumerate}

\section{}
در این قسمت با نحوه ارجاع به سایر منابع آشنا می\,شوید:\\
\indent
به صفحه درس سیستم عامل دکتر محمّدرضا حیدرپور ارجاع داده می\,شود \cite{b1}.

\section{ضمیمه}
برای آشنایی بیشتر با \lr{\LaTeX}، با جست\,و\,جو در اینترنت منابع مفیدی خواهید یافت.


\section*{منابع}
\renewcommand{\section}[2]{}%
\begin{thebibliography}{99} % assumes less than 100 references
%چنانچه مرجع فارسی نیز داشته باشید باید دستور فوق را فعال کنید و مراجع فارسی خود را بعد از این دستور وارد کنید


\begin{LTRitems}

\resetlatinfont

\bibitem{b1} http://mrheidar.ir/courses/operating\_system.html
\end{LTRitems}

\end{thebibliography}


\end{document}

\documentclass{article}

\usepackage{graphicx}
\usepackage{rotating}
\usepackage{amsmath}
\usepackage{amssymb}
\usepackage{fancyhdr}
\usepackage{listings}
\usepackage{xcolor}
\usepackage{color}
\usepackage{amsfonts}
\usepackage{textcomp}
\usepackage{float}
\usepackage[sorting=none]{biblatex}
\usepackage[margin=1in]{geometry}
\usepackage[font={small,it}]{caption}
\usepackage{placeins}
\usepackage{xepersian}

%\DeclareMathOperator*{\btie}{\bowtie}
\addbibresource{bibliography.bib}
\settextfont[Scale=1.2]{B-NAZANIN.TTF}
\setlatintextfont[Scale=1]{Times New Roman}
\renewcommand{\baselinestretch}{1.5}
\pagestyle{fancy}
\fancyhf{}
\rhead{تکلیف اول درس مبانی یادگیری ماشین (بخش تئوری)}
\lhead{\thepage}
\rfoot{علیرضا ابره فروش}
\lfoot{9816603}
\renewcommand{\headrulewidth}{1pt}
\renewcommand{\footrulewidth}{1pt}
\newcommand{\Lagr}{\mathcal{L}}
%%%%%%%%%%
\lstset
{
    language=[latex]tex,
    basicstyle=\ttfamily,
    commentstyle=\color{black},
    columns=fullflexible,
    keepspaces=true,
    upquote=true,
    showstringspaces=false,
    morestring=[s]\\\%,
    stringstyle=\color{black},
}
%%%%%%%%%%
%beginMatlab
\definecolor{mygreen}{RGB}{28,172,0} % color values Red, Green, Blue
\definecolor{mylilas}{RGB}{170,55,241}
%endMatlab
\begin{document}
%beginMatlab
\lstset{language=Matlab,%
    %basicstyle=\color{red},
    breaklines=true,%
    morekeywords={matlab2tikz},
    keywordstyle=\color{blue},%
    morekeywords=[2]{1}, keywordstyle=[2]{\color{black}},
    identifierstyle=\color{black},%
    stringstyle=\color{mylilas},
    commentstyle=\color{mygreen},%
    showstringspaces=false,%without this there will be a symbol in the places where there is a space
    numbers=left,%
    numberstyle={\tiny \color{black}},% size of the numbers
    numbersep=9pt, % this defines how far the numbers are from the text
    emph=[1]{for,end,break},emphstyle=[1]\color{red}, %some words to emphasise
    %emph=[2]{word1,word2}, emphstyle=[2]{style},    
}
%endMatlab
\begin{titlepage}
\begin{center}
\includegraphics[width=0.4\textwidth]{IUT Logo.png}\\
        
\LARGE
\textbf{دانشگاه صنعتی اصفهان}\\
\textbf{دانشکده مهندسی برق و کامپیوتر}\\
        
\vfill
        
\huge
\textbf{عنوان: تکلیف اول درس سیستم‌های عامل 1}\\
        
\vfill
        
\LARGE
\textbf{نام و نام خانوادگی: علیرضا ابره فروش}\\
\textbf{شماره دانشجویی: 9816603}\\
\textbf{نیم\,سال تحصیلی: پاییز 1400}\\
\textbf{مدرّس: دکتر محمّدرضا حیدرپور}\\
\textbf{دستیاران آموزشی: مجید فرهادی - دانیال مهرآیین - محمّد نعیمی}\\
\end{center}
\end{titlepage}


%\tableofcontents
\newpage


%1
\section{}
\subsection{الف}
\begin{itemize}
  \item مسیر آبی که نسبتا مستقیم است مربوط به روش \lr{Batch GD} است. چون در هر گام کل مجموعه‌ی داده‌ها را می‌بیند.
  \item مسیر قرمز که نوسان بیشتری نسبت به مسیر آبی و نوسان کمتری نسبت به مسیر سبز دارد مربوط به روش \lr{Mini-batch GD} است. چون در هر گام بخشی از مجموعه‌ی داده‌ها را می‌بیند.
  \item مسیر سبز که در مقایسه با بقیه مسیرها بیشترین نوسان را دارد مربوط به روش \lr{SGD} است. چون در هر گام یک داده از مجموعه داده‌ها را می‌بیند.
\end{itemize}


\subsection{ب}
داده‌های تصویر \lr{a} برخلاف داده‌های تصویر \lr{b} نرمال نشده‌اند. در داده‌هایی که \lr{feature}ها \lr{scale} نشده‌اند، تغییر مقادیر پارامترها باعث بروز اعوجاج شدیدتر نسبت به داده‌های نرمال شده می‌شود و در نتیجه همگرایی مدل را کند می‌کند.


%2
\section{}
\subsection{الف}
\lr{MAE} بهتر است. چون تعداد داده‌های \lr{outlier} قابل توجه است، در صورت استفاده از \lr{MSE}، توان دوی \lr{feature}های این داده‌ها اثر داده می‌شوند و اثر سایر داده‌ها (که \lr{outlier} نیستند و مدل را مطلوب‌تر آموزش می‌دهند) تقریبا ناچیز می‌شود.

\subsection{ب}
\lr{MSE} بهتر است. چون احتمالا مدل دچار \lr{overfitting} شده است، مقادیر پیشبینی شده به مقادیر واقعی خیلی نزدیک‌اند و \lr{MSE} آن‌ها را بیشتر بازتاب می‌دهد.

\subsection{ج}
\lr{MAE} بهتر است. از آنجایی که داده‌ها مقیاس‌های متفاوتی دارند، اگر از \lr{MSE} استفاده کنیم اثر نرمال نبودن داده‌ها به طور نمایی در مدل تاثیر می‌گذارد.

%3
\section{}
ستون \lr{A} که به نسبت سایر ستون‌ها ضرایب بزرگتری دارد فاقد \lr{regularization term} است. پس مربوط به تابع هزینه‌ی اول است.\\
ستون \lr{B} فاقد ضرایب صفر است. می‌دانیم که امکان صفر شدن ضرایب در \lr{lasso regression} موجود است. پس مربوط به تابع دوم است.\\
ستون \lr{C} دارای ضرایب صفر است. می‌دانیم که در \lr{ridge regression} هرگز ضرایب صفر نمی‌شوند. پس مربوط به تابع سوم است.

%4
\section{}
\begin{latin}
$
p\left( x_k | \theta \right) = \sqrt{\theta}x_k^{\sqrt{\theta} - 1}\\
p\left( X | \theta \right) = \prod_{k = 1} ^ {n} p\left( x_k | \theta \right) \\ \\ \\
\ln \left( p\left( X | \theta \right) \right) = \sum_{k = 1} ^ {n}\ln \left( p\left( x_k | \theta \right) \right) = \sum_{k = 1} ^ {n} \ln \left( \sqrt{\theta}x_k^{\sqrt{\theta} - 1} \right) = \sum_{k = 1} ^ {n} \left[ \frac{1}{2}\ln \left( \theta \right) + \left( \sqrt{\theta} - 1 \right) \ln \left( x_k \right) \right]\\ \\ \\
\frac{\partial }{\partial \theta} \ln \left( p\left( X | \theta \right) \right) = \frac{n}{2\theta} - \frac{\sum_{k = 1}^{n} \ln \left( x_k \right)}{2\sqrt{\theta}} = 0 \\ \\
\theta = \left( \frac{n}{\sum_{k = 1}^{n} \ln \left( x_k \right)} \right) ^ 2
$
\end{latin}


%5
\section{}
$
p \left( \mu | X \right) \propto p \left( x_k | \mu \right) . p \left( \mu \right) \\ \\ \\
p \left( \mu | X \right) = \left[ \prod_{k = 1} ^ {n} \frac{1}{\sqrt{2\pi\sigma ^ {'2}}}e ^ {-\frac{\left( x_k-\mu \right) ^ 2}{2\sigma ^ {'2}}} \right] . \frac{1}{\left(2\pi \right) ^ \frac{l}{2}\sigma_\mu^2}e ^ {-\frac{\left\| \mu - \mu_0\right\| ^ 2}{2\sigma_\mu^2}} \\ \\ \\
\ln \left( p \left( \mu | X \right) \right) = \sum_{k = 1} ^ {n} \left[ -\ln \left( \sqrt{2\pi\sigma ^ {'2}} \right)-\frac{\left( x_k-\mu \right) ^ 2}{2\sigma ^ {'2}}  \right] - \ln \left( \left( 2\pi\right) ^ \frac{l}{2}\sigma_\mu^2 \right) - \frac{\left\| \mu - \mu_0\right\| ^ 2}{2\sigma_\mu^2}\\ \\ \\
\frac{\partial }{\partial \mu} \ln \left( p \left( \mu | X \right) \right) = 0 \\ \\
\sum_{k = 1}^{n}\frac{x_k-\mu}{\sigma ^ {'2}} = \frac{\left\| \mu - \mu_0 \right\|}{2\sigma_\mu^2}\\ \\
\mu = \frac{\frac{\sum_{k = 1}^{n}x_k}{\sigma ^ {'2}} + \frac{\mu}{2\sigma_\mu^2}}{n+\frac{1}{2\sigma_\mu^2}}
$




%6
\section{}
$\Lagr (w)$: تابع هزینه\\
$\Lagr_i (w)$: هزینه‌ی \lr{training example}ِ $i$ام\\
$w^t$: وزن‌ها در گام $t$ام\\
\begin{latin}
$\\
y =
\begin{bmatrix}
y_1 \\
y_2 \\
\vdots \\
y_9
\end{bmatrix},
X =
\begin{bmatrix}
(X_1)^2 & X_1 & 1 \\
(X_2)^2 & X_2 & 1 \\
\vdots  & \vdots  & \vdots  \\
(X_9)^2 & X_9 & 1
\end{bmatrix},
w =
\begin{bmatrix}
a \\
b \\
c
\end{bmatrix}\\ \\
\Lagr_i (w ^ {(t)}) = \frac{1}{2}\left( y_i - X_i ^ T w ^ {(t)} \right) ^ 2\\
\nabla \Lagr_i (w ^ {(t)}) = -X_i\left( y_i - X_i^Tw^{(t)} \right)\\
w ^ {(t + 1)} = w ^ {(t)} - \alpha \nabla \Lagr_i (w ^ {(t)})\\
w ^ {(0)} = 
\begin{bmatrix}
0 \\
0 \\
0
\end{bmatrix}\\
w ^ {(1)} =%%%%%
\begin{bmatrix}
0.00000000e+00 \\
0.00000000e+00 \\
0.00000000e+00
\end{bmatrix}
-0.1 \times \left( -
\begin{bmatrix}
35.38^2 \\
35.38 \\
1
\end{bmatrix}
\left(
2955.53 -
\begin{bmatrix}
35.38 ^ 2 & 35.38 & 1
\end{bmatrix}
\begin{bmatrix}
0.00000000e+00 \\
0.00000000e+00 \\
0.00000000e+00
\end{bmatrix}
\right)
\right)=
\begin{bmatrix}
3.69956813e+05 \\
1.04566651e+04 \\
2.95553000e+02
\end{bmatrix}\\
w ^ {(2)} =%%%%%
\begin{bmatrix}
3.69956813e+05 \\
1.04566651e+04 \\
2.95553000e+02
\end{bmatrix}
-0.1 \times \left( -
\begin{bmatrix}
15.32^2 \\
15.32 \\
1
\end{bmatrix}
\left(
560.30 -
\begin{bmatrix}
15.32 ^ 2 & 15.32 & 1
\end{bmatrix}
\begin{bmatrix}
3.69956813e+05 \\
1.04566651e+04 \\
2.95553000e+02
\end{bmatrix}
\right)
\right)=
\begin{bmatrix}
-2.04129879e+09 \\
-1.33257738e+08 \\
-8.69867277e+06
\end{bmatrix}\\
w ^ {(3)} =%%%%%
\begin{bmatrix}
-2.04129879e+09 \\
-1.33257738e+08 \\
-8.69867277e+06
\end{bmatrix}
-0.1 \times \left( -
\begin{bmatrix}
11.74^2 \\
11.74 \\
1
\end{bmatrix}
\left(
334.32 -
\begin{bmatrix}
11.74 ^ 2 & 11.74 & 1
\end{bmatrix}
\begin{bmatrix}
-2.04129879e+09 \\
-1.33257738e+08 \\
-8.69867277e+06
\end{bmatrix}
\right)
\right)=
\begin{bmatrix}
3.89738346e+12 \\
3.32015359e+11 \\
2.82833471e+10
\end{bmatrix}\\
w ^ {(4)} =%%%%%
\begin{bmatrix}
3.89738346e+12 \\
3.32015359e+11 \\
2.82833471e+10
\end{bmatrix}
-0.1 \times \left( -
\begin{bmatrix}
19.05^2 \\
19.05 \\
1
\end{bmatrix}
\left(
864.44 -
\begin{bmatrix}
19.05 ^ 2 & 19.05 & 1
\end{bmatrix}
\begin{bmatrix}
3.89738346e+12 \\
3.32015359e+11 \\
2.82833471e+10
\end{bmatrix}
\right)
\right)=
\begin{bmatrix}
-5.15545092e+16 \\
-2.70614602e+15 \\
-1.42044054e+14
\end{bmatrix}\\
w ^ {(5)} =%%%%%
\begin{bmatrix}
-5.15545092e+16 \\
-2.70614602e+15 \\
-1.42044054e+14
\end{bmatrix}
-0.1 \times \left( -
\begin{bmatrix}
26.85^2 \\
26.85 \\
1
\end{bmatrix}
\left(
1709.09 -
\begin{bmatrix}
26.85 ^ 2 & 26.85 & 1
\end{bmatrix}
\begin{bmatrix}
-5.15545092e+16 \\
-2.70614602e+15 \\
-1.42044054e+14
\end{bmatrix}
\right)
\right)=
\begin{bmatrix}
2.68463555e+21 \\
9.99856406e+19 \\
3.72381873e+18
\end{bmatrix}\\
w ^ {(6)} =%%%%%
\begin{bmatrix}
2.68463555e+21 \\
9.99856406e+19 \\
3.72381873e+18
\end{bmatrix}
-0.1 \times \left( -
\begin{bmatrix}
39.45^2 \\
39.45 \\
1
\end{bmatrix}
\left(
3670.48 -
\begin{bmatrix}
39.45 ^ 2 & 39.45 & 1
\end{bmatrix}
\begin{bmatrix}
2.68463555e+21 \\
9.99856406e+19 \\
3.72381873e+18
\end{bmatrix}
\right)
\right)=
\begin{bmatrix}
-6.50851298e+26 \\
-1.64980998e+25 \\
-4.18201594e+23
\end{bmatrix}\\
w ^ {(7)} =%%%%%
\begin{bmatrix}
-6.50851298e+26 \\
-1.64980998e+25 \\
-4.18201594e+23
\end{bmatrix}
-0.1 \times \left( -
\begin{bmatrix}
30.51^2 \\
30.51 \\
1
\end{bmatrix}
\left(
2202.93 -
\begin{bmatrix}
30.51 ^ 2 & 30.51 & 1
\end{bmatrix}
\begin{bmatrix}
-6.50851298e+26 \\
-1.64980998e+25 \\
-4.18201594e+23
\end{bmatrix}
\right)
\right)=
\begin{bmatrix}
5.64425427e+31 \\
1.84997346e+30 \\
6.06351097e+28
\end{bmatrix}\\
w ^ {(8)} =%%%%%
\begin{bmatrix}
5.64425427e+31 \\
1.84997346e+30 \\
6.06351097e+28
\end{bmatrix}
-0.1 \times \left( -
\begin{bmatrix}
3.98^2 \\
3.98 \\
1
\end{bmatrix}
\left(
13.08 -
\begin{bmatrix}
30.51 ^ 2 & 30.51 & 1
\end{bmatrix}
\begin{bmatrix}
5.64425427e+31 \\
1.84997346e+30 \\
6.06351097e+28
\end{bmatrix}
\right)
\right)=
\begin{bmatrix}
-1.37156316e+33 \\
-3.56945428e+32 \\
-9.00889632e+31
\end{bmatrix}\\
w ^ {(9)} =%%%%%
\begin{bmatrix}
-1.37156316e+33 \\
-3.56945428e+32 \\
-9.00889632e+31
\end{bmatrix}
-0.1 \times \left( -
\begin{bmatrix}
0.29^2 \\
0.29 \\
1
\end{bmatrix}
\left(
2.28 -
\begin{bmatrix}
0.29 ^ 2 & 0.29 & 1
\end{bmatrix}
\begin{bmatrix}
-1.37156316e+33 \\
-3.56945428e+32 \\
-9.00889632e+31
\end{bmatrix}
\right)
\right)=
\begin{bmatrix}
-1.36896487e+33 \\
-3.47985832e+32 \\
-5.91938034e+31
\end{bmatrix}\\
$
\end{latin}


در هر گام از بین \lr{training example}ها یکی را به صورت تصادفی انتخاب می‌کنیم. این کار را به تعداد \lr{training example}ها تکرار می‌کنیم تا کل داده‌ها توسط مدل دیده شوند.




%%%%%%%%%%%%%%%%%%%%%%%%%%%%%%%%%%%
%%%%%%%%%%%%%%%%%%%%%%%%%%%%%%%%%%%
%%%%%%%%%%%%%%%%%%%%%%%%%%%%%%%%%%%






\section*{منابع}
\renewcommand{\section}[2]{}%
\begin{thebibliography}{99} % assumes less than 100 references
%چنانچه مرجع فارسی نیز داشته باشید باید دستور فوق را فعال کنید و مراجع فارسی خود را بعد از این دستور وارد کنید


\begin{LTRitems}

\resetlatinfont

\bibitem{b1}
\end{LTRitems}

\end{thebibliography}


\end{document}

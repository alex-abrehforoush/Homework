\documentclass{article}

\usepackage{graphicx}
\usepackage{rotating}
\usepackage{amsmath}
\usepackage{fancyhdr}
\usepackage{listings}
\usepackage{xcolor}
\usepackage{color}
\usepackage{textcomp}
\usepackage{float}
\usepackage[sorting=none]{biblatex}
\usepackage[margin=1in]{geometry}
\usepackage[font={small,it}]{caption}
\usepackage{placeins}
\usepackage{xepersian}

%\DeclareMathOperator*{\btie}{\bowtie}
\addbibresource{bibliography.bib}
\settextfont[Scale=1.2]{B-NAZANIN.TTF}
\setlatintextfont[Scale=1]{Times New Roman}
\renewcommand{\baselinestretch}{1.5}
\pagestyle{fancy}
\fancyhf{}
\rhead{تکلیف پنجم درس شبکه‌های کامپیوتری 2}
\lhead{\thepage}
\rfoot{علیرضا ابره فروش}
\lfoot{9816603}
\renewcommand{\headrulewidth}{1pt}
\renewcommand{\footrulewidth}{1pt}
%%%%%%%%%%
\lstset
{
    language=[latex]tex,
    basicstyle=\ttfamily,
    commentstyle=\color{black},
    columns=fullflexible,
    keepspaces=true,
    upquote=true,
    showstringspaces=false,
    morestring=[s]\\\%,
    stringstyle=\color{black},
}
%%%%%%%%%%
%beginMatlab
\definecolor{mygreen}{RGB}{28,172,0} % color values Red, Green, Blue
\definecolor{mylilas}{RGB}{170,55,241}
%endMatlab
\begin{document}
%beginMatlab
\lstset{language=Matlab,%
    %basicstyle=\color{red},
    breaklines=true,%
    morekeywords={matlab2tikz},
    keywordstyle=\color{blue},%
    morekeywords=[2]{1}, keywordstyle=[2]{\color{black}},
    identifierstyle=\color{black},%
    stringstyle=\color{mylilas},
    commentstyle=\color{mygreen},%
    showstringspaces=false,%without this there will be a symbol in the places where there is a space
    numbers=left,%
    numberstyle={\tiny \color{black}},% size of the numbers
    numbersep=9pt, % this defines how far the numbers are from the text
    emph=[1]{for,end,break},emphstyle=[1]\color{red}, %some words to emphasise
    %emph=[2]{word1,word2}, emphstyle=[2]{style},    
}
%endMatlab
\begin{titlepage}
\begin{center}
\includegraphics[width=0.4\textwidth]{IUT Logo.png}\\
        
\LARGE
\textbf{دانشگاه صنعتی اصفهان}\\
\textbf{دانشکده مهندسی برق و کامپیوتر}\\
        
\vfill
        
\huge
\textbf{عنوان: تکلیف اول درس سیستم‌های عامل 1}\\
        
\vfill
        
\LARGE
\textbf{نام و نام خانوادگی: علیرضا ابره فروش}\\
\textbf{شماره دانشجویی: 9816603}\\
\textbf{نیم\,سال تحصیلی: پاییز 1400}\\
\textbf{مدرّس: دکتر محمّدرضا حیدرپور}\\
\textbf{دستیاران آموزشی: مجید فرهادی - دانیال مهرآیین - محمّد نعیمی}\\
\end{center}
\end{titlepage}


%\tableofcontents
\newpage

\section{فصل 1}
\subsection{سوال 26}
\subsubsection{\lr{a}}
\begin{latin}
$
d_{prop}=20000 \: kilometers\times \frac{1}{2.5\times 10^{8} \: meters/sec}=0.08 \: sec
\\
R.d_{prop}=1\: Gbps\times 0.08 \: sec=8\times10^{7}\: bit
$
\end{latin}
\subsubsection{\lr{b}}
\begin{latin}
$
min(8\times10^{7}\: bit, \: 800000\:bit)
$
\end{latin}
\subsubsection{\lr{c}}
\begin{latin}
$
width=\frac{20000\: kilometers}{8\times 10^{7}\: bit}=0.25\: meters/bit
$
\end{latin}

\subsection{سوال 27}
از آنجایی که سرعت لینک بین سرور و روتر دو برابر سرعت لینک بین روتر و کلاینت است، از هر دو پیامی که در واحد زمان به روتر می‌رسد، یکی از روتر به کلاینت می‌رسد و دومی بافر می‌شود. در نتیجه برای بافر شدن 4 پیام و دراپ شدن پیام بعدی، باید 9 پیام ارسال شده باشد که پیام 9ام دراپ می‌شود.

\section{فصل 3}
\subsection{سوال 28}
از آنجایی که سرعت لینک \lr{100 Mbps} است، سرعت ارسال هاست \lr{A} در حالت بیشینه \lr{100 Mbps} می‌تواند باشد. سرعت پر شدن بافر از سرعت خالی شدن آن بیشتر است و در واقع بافر با نرخ
$
100 \: Mbps - 50 \: Mbps = 50 \: Mbps
$
پر می‌شود. هنگامی که بافر پر می‌شود، هاست \lr{B} مقدار \lr{RcvWindow} را برابر با صفر قرار می‌دهد که با این کار به هاست \lr{A} پیغام می‌دهد که بافر پر شده است. هاست \lr{B}، ارسال اطلاعات را تا زمانی که سگمنتی با \lr{RcvWindow} بزرگ‌تر از صفر دریافت کند، متوقف می‌کند. در واقع هاست \lr{B}، بر اساس مقادیر \lr{RcvWindow}ی که هاست \lr{B} تعیین می‌کند ارسال یا توقف بسته را انجام می‌دهد. سرعت نهایی (در بلند مدت) ارسال داده از \lr{A} به \lr{B} به طور متوسط بیشتر از
$
50 \: Mbps
$
نخواهد بود.

%%%%%%%%%%%%%%%%%%%%%%%%%%%%%%%%%%%
%%%%%%%%%%%%%%%%%%%%%%%%%%%%%%%%%%%
%%%%%%%%%%%%%%%%%%%%%%%%%%%%%%%%%%%

\section*{منابع}
\renewcommand{\section}[2]{}%
\begin{thebibliography}{99} % assumes less than 100 references
%چنانچه مرجع فارسی نیز داشته باشید باید دستور فوق را فعال کنید و مراجع فارسی خود را بعد از این دستور وارد کنید


\begin{LTRitems}

\resetlatinfont

\bibitem{b1}
\end{LTRitems}

\end{thebibliography}


\end{document}

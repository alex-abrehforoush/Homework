\documentclass{article}

\usepackage{graphicx}
\usepackage{rotating}
\usepackage{amsmath}
\usepackage{fancyhdr}
\usepackage{listings}
\usepackage{xcolor}
\usepackage{color}
\usepackage{textcomp}
\usepackage{float}
\usepackage[sorting=none]{biblatex}
\usepackage[margin=1in]{geometry}
\usepackage[font={small,it}]{caption}
\usepackage{placeins}
\usepackage{xepersian}

%\DeclareMathOperator*{\btie}{\bowtie}
\addbibresource{bibliography.bib}
\settextfont[Scale=1.2]{B-NAZANIN.TTF}
\setlatintextfont[Scale=1]{Times New Roman}
\renewcommand{\baselinestretch}{1.5}
\pagestyle{fancy}
\fancyhf{}
\rhead{تکلیف سوم درس شبکه‌های کامپیوتری 2}
\lhead{\thepage}
\rfoot{علیرضا ابره فروش}
\lfoot{9816603}
\renewcommand{\headrulewidth}{1pt}
\renewcommand{\footrulewidth}{1pt}
%%%%%%%%%%
\lstset
{
    language=[latex]tex,
    basicstyle=\ttfamily,
    commentstyle=\color{black},
    columns=fullflexible,
    keepspaces=true,
    upquote=true,
    showstringspaces=false,
    morestring=[s]\\\%,
    stringstyle=\color{black},
}
%%%%%%%%%%
%beginMatlab
\definecolor{mygreen}{RGB}{28,172,0} % color values Red, Green, Blue
\definecolor{mylilas}{RGB}{170,55,241}
%endMatlab
\begin{document}
%beginMatlab
\lstset{language=Matlab,%
    %basicstyle=\color{red},
    breaklines=true,%
    morekeywords={matlab2tikz},
    keywordstyle=\color{blue},%
    morekeywords=[2]{1}, keywordstyle=[2]{\color{black}},
    identifierstyle=\color{black},%
    stringstyle=\color{mylilas},
    commentstyle=\color{mygreen},%
    showstringspaces=false,%without this there will be a symbol in the places where there is a space
    numbers=left,%
    numberstyle={\tiny \color{black}},% size of the numbers
    numbersep=9pt, % this defines how far the numbers are from the text
    emph=[1]{for,end,break},emphstyle=[1]\color{red}, %some words to emphasise
    %emph=[2]{word1,word2}, emphstyle=[2]{style},    
}
%endMatlab
\begin{titlepage}
\begin{center}
\includegraphics[width=0.4\textwidth]{IUT Logo.png}\\
        
\LARGE
\textbf{دانشگاه صنعتی اصفهان}\\
\textbf{دانشکده مهندسی برق و کامپیوتر}\\
        
\vfill
        
\huge
\textbf{عنوان: تکلیف اول درس سیستم‌های عامل 1}\\
        
\vfill
        
\LARGE
\textbf{نام و نام خانوادگی: علیرضا ابره فروش}\\
\textbf{شماره دانشجویی: 9816603}\\
\textbf{نیم\,سال تحصیلی: پاییز 1400}\\
\textbf{مدرّس: دکتر محمّدرضا حیدرپور}\\
\textbf{دستیاران آموزشی: مجید فرهادی - دانیال مهرآیین - محمّد نعیمی}\\
\end{center}
\end{titlepage}


%\tableofcontents
\newpage


\section{\lr{Mininet}}%1
\subsection{معرفی}
\lr{Mininet} یک شبیه‌ساز شبکه است که قادر است شبکه‌ای از \lr{Host}ها، \lr{Switch}ها، \lr{Controller}ها و \lr{Link}ها بسازد. هاست‌های \lr{Mininet} نرم‌افزار شبکه‌ی لینوکس استاندارد را اجرا می‌کنند و سوئیچ‌های آن برای مسیریابیِ انعطاف‌پذیرِ \lr{Custom} از \lr{OpenFlow} پشتیبانی می‌کنند. همچنین \lr{Mininet} جهت تحقیق، توسعه، آموزش، پروتوتایپینگ، تست، دیباگ و یا هر عمل دیگری که در آن از وجود یک شبکه‌ی کاملا آزمایشگاهی بهره‌گیری می‌شود، به کار می‌رود.
\subsection{قابلیت‌ها و توانایی‌ها}
\begin{itemize}
    \item [$\bullet$] \lr{Mininet} یک بستر شبکه‌ی ساده و ارزان جهت توسعه‌ی اپلیکیشن‌های \lr{OpenFlow} ارائه می‌دهد.
    \item [$\bullet$] \lr{Mininet} به توسعه‌دهندگان امکان این را می‌دهد که به صورت موازی و مستقل بر روی یک توپولوژی یکسان کار کنند.
    \item [$\bullet$] \lr{Mininet} از \lr{system-level regression tests} که قابلیت تکرار و بسته‌بندی ساده را دارند پشتیبانی می‌کند.
    \item [$\bullet$] \lr{Mininet} امکان تست توپولوژی‌های پیچیده بدون نیاز به سیم‌کشی فیزیکی شبکه را می‌دهد.
    \item [$\bullet$] \lr{Mininet} دارای یک \lr{CLI}ِ خاص است که با توپولوژی و \lr{OpenFlow} آشنایی دارد و برای دیباگ یا اجرای تست‌های سرتاسری شبکه مناسب است.
    \item [$\bullet$] \lr{Mininet} از \lr{arbitrary custom topologies} پشتیبانی می‌کند و دارای مجموعه‌ای پایه از توپولوژی‌های \lr{parametrized} است.
    \item [$\bullet$] \lr{Mininet} قابل استفاده بدون نیاز به برنامه‌نویسی و به صورت \lr{out of the box} است درحالی که یک اِی‌پی‌آیِ روان و قابل توسعه در بستر پایتون برای ساخت شبکه و آزمایش آن ارائه می‌دهد.

\end{itemize}

\subsection{نحوه‌ی نصب}
جهت نصب این نرم‌افزار در لینوکس به ترتیب زیر عمل می‌کنیم:
\newline
ابتدا گیت را نصب می‌کنیم و سپس مخزن شامل نرم‌افزار در گیت‌هاب را کلون می‌کنیم.
\begin{latin}
\$ sudo apt install git
\newline
\$ sudo git clone https://github.com/mininet/mininet
\end{latin}
سپس وارد پوشه‌ی \lr{mininet/util} شده و به ترتیب زیر نرم‌افزار را نصب می‌کنیم.
\begin{latin}
\$ ./install.sh -a
\newline
\$ sudo mn --test pingall
\end{latin}
%%%%%%%%%%%%%%%%%%%%%%%%%%%%%%%%%%%
%%%%%%%%%%%%%%%%%%%%%%%%%%%%%%%%%%%
%%%%%%%%%%%%%%%%%%%%%%%%%%%%%%%%%%%

\section*{منابع}
\renewcommand{\section}[2]{}%
\begin{thebibliography}{99} % assumes less than 100 references
%چنانچه مرجع فارسی نیز داشته باشید باید دستور فوق را فعال کنید و مراجع فارسی خود را بعد از این دستور وارد کنید


\begin{LTRitems}

\resetlatinfont

\bibitem{b1} http://mininet.org/overview/
\end{LTRitems}

\end{thebibliography}


\end{document}

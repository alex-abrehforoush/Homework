\documentclass{article}

\usepackage{graphicx}
\usepackage{rotating}
\usepackage{amsmath}
\usepackage{amssymb}
\usepackage{mathrsfs}
\usepackage{amsthm}
\usepackage{fancyhdr}
\usepackage{listings}
\usepackage{xcolor}
\usepackage{color}
\usepackage{amsfonts}
\usepackage{textcomp}
\usepackage{float}
\usepackage{neuralnetwork}
\usepackage{pgfplots}
\usepackage[sorting=none]{biblatex}
\usepackage[margin=1in]{geometry}
\usepackage[font={small,it}]{caption}
\usepackage{placeins}
\usepackage{xepersian}

\pgfplotsset{width=8cm,compat=1.17}

%\DeclareMathOperator*{\btie}{\bowtie}
\addbibresource{bibliography.bib}
\settextfont[Scale=1.2]{B-NAZANIN.TTF}
\setlatintextfont[Scale=1]{Times New Roman}
\renewcommand{\baselinestretch}{1.5}
\pagestyle{fancy}
\fancyhf{}
\rhead{تکلیف اول درس پیچیدگی محاسباتی}
\lhead{\thepage}
\rfoot{علیرضا ابره فروش}
\lfoot{9816603}
\renewcommand{\headrulewidth}{1pt}
\renewcommand{\footrulewidth}{1pt}
\newcommand{\Lagr}{\mathcal{L}}
\newcommand{\Mod}[1]{\ (\mathrm{mod}\ #1)}
%%%%%%%%%%
\lstset
{
    language=[latex]tex,
    basicstyle=\ttfamily,
    commentstyle=\color{black},
    columns=fullflexible,
    keepspaces=true,
    upquote=true,
    showstringspaces=false,
    morestring=[s]\\\%,
    stringstyle=\color{black},
}
%%%%%%%%%%
%beginMatlab
\definecolor{mygreen}{RGB}{28,172,0} % color values Red, Green, Blue
\definecolor{mylilas}{RGB}{170,55,241}
%endMatlab
\begin{document}
%beginMatlab
\lstset{language=Matlab,%
    %basicstyle=\color{red},
    breaklines=true,%
    morekeywords={matlab2tikz},
    keywordstyle=\color{blue},%
    morekeywords=[2]{1}, keywordstyle=[2]{\color{black}},
    identifierstyle=\color{black},%
    stringstyle=\color{mylilas},
    commentstyle=\color{mygreen},%
    showstringspaces=false,%without this there will be a symbol in the places where there is a space
    numbers=left,%
    numberstyle={\tiny \color{black}},% size of the numbers
    numbersep=9pt, % this defines how far the numbers are from the text
    emph=[1]{for,end,break},emphstyle=[1]\color{red}, %some words to emphasise
    %emph=[2]{word1,word2}, emphstyle=[2]{style},    
}
%endMatlab
\begin{titlepage}
\begin{center}
\includegraphics[width=0.4\textwidth]{IUT Logo.png}\\
        
\LARGE
\textbf{دانشگاه صنعتی اصفهان}\\
\textbf{دانشکده مهندسی برق و کامپیوتر}\\
        
\vfill
        
\huge
\textbf{عنوان: تکلیف اول درس سیستم‌های عامل 1}\\
        
\vfill
        
\LARGE
\textbf{نام و نام خانوادگی: علیرضا ابره فروش}\\
\textbf{شماره دانشجویی: 9816603}\\
\textbf{نیم\,سال تحصیلی: پاییز 1400}\\
\textbf{مدرّس: دکتر محمّدرضا حیدرپور}\\
\textbf{دستیاران آموزشی: مجید فرهادی - دانیال مهرآیین - محمّد نعیمی}\\
\end{center}
\end{titlepage}


%\tableofcontents
\newpage


%1
\section{}

\begin{latin}
$
c_1.n^{2^{n}}.2^{c_2n.c_3^{n}} \in 2 ^ {2 ^ {O\left( n \right)}} \Longleftrightarrow \\
\log \left( c_1 \right) + 2 ^ n\log \left( n \right) + c_2 n . c_3 ^ n \in 2 ^ {O\left( n \right)} 
\overset{c_3 > 2}{\Longleftrightarrow} \\
n . c_3 ^ n \in 2 ^ {O\left( n \right)} 
\Longleftrightarrow \\
\log \left( n \right) + n \log \left( c_3 \right) \in O\left( n \right) \Longleftrightarrow \\
n \in O\left( n \right) \\
\qedsymbol{}
$
\end{latin}

%2
\section{}
\begin{latin}
Let $A$ and $B$ be two languages in $P$. We aim to prove that there exists a polynomial-time reduction from $A$ to $B$. 

Since $A$ is in $P$, there exists a deterministic polynomial-time Turing machine $M_A$ that decides $A$. Similarly, since $B$ is in $P$, there exists a deterministic polynomial-time Turing machine $M_B$ that decides $B$.

Given a string $z$, we can check whether $z$ is in $A$ using $M_A$ in polynomial time.

Now, let's define our reduction function $f$ as follows:

Given an input string $z$:
\begin{itemize}
    \item If $z$ is in $A$, output $x$ (since $z$ is in $A$, $f(z)$ should be in $B$).
    \item If $z$ is not in $A$, output $y$ (since $z$ is not in $A$, $f(z)$ should not be in $B$).
\end{itemize}

Since $M_B$ runs in polynomial time, and the construction of $f(z)$ also runs in polynomial time, the function $f$ is a polynomial-time computable function.

Therefore, we've shown that there exists a polynomial-time reduction from $A$ to $B$.
\end{latin}


%3
\section{}
\subsection{}
\begin{latin}
Assume, for the sake of contradiction, that $B_1$ is decidable. 

If $B_1$ is decidable, then there exists a Turing machine $M_{B_1}$ that decides it. 

We can use $M_{B_1}$ to decide $A_1$ as follows:

Given an input $< M >$, where $M$ is a Turing machine:
\begin{enumerate}
    \item Construct a Turing machine $N$ such that $L(N)$ is the complement of $L(M)$, i.e., $L(N) = \Sigma^* \setminus L(M)$.
    \item Encode $< M >$ and the description of $N$ as input for $M_{B_1}$.
    \item If $M_{B_1}$ accepts, output "accept". If $M_{B_1}$ rejects, output "reject".
\end{enumerate}

If $M_{B_1}$ accepts, it means that $N$ accepts all strings, i.e., $L(N) = \Sigma^*$. This implies that $L(M)$ is empty, and thus $< M > \in A_1$. 

If $M_{B_1}$ rejects, it means that $N$ does not accept all strings, i.e., $L(N) \neq \Sigma^*$. This implies that $L(M)$ is not empty, and thus $< M > \notin A_1$. 

However, we know that $A_1$ is undecidable, which contradicts our assumption that $B_1$ is decidable.

Therefore, our initial assumption that $B_1$ is decidable must be false. Hence, $B_1$ is undecidable.
\end{latin}


\subsection{}


%4
\section{}
\begin{latin}
Use the construction: The TM $M'$ encodes the $k$ tapes of $M$ (including its input and output tapes) on a single tape by using locations $1, k + 1, 2k + 1, \cdots$ to encode the first tape, locations $2, k + 2, 2k + 2, \cdots$ to encode the second tape etc. For every symbol $a$ in $M$’s alphabet, $M'$ will contain both the symbol $a$ and the symbol $\hat{a}$. In the encoding of each tape, exactly one symbol will be of the "hat type," indicating that the corresponding head of $M$ is positioned in that location. $M'$ uses the input and output tape in the same way $M$ does. To simulate one step of $M$, the machine $M'$ makes two sweeps of its work tape: first it sweeps
the tape in the left-to-right direction and records to its state register the $k$
symbols that are hatted. Then $M'$ uses $M$’s transition function to determine
the new state, symbols, and head movements and sweeps the tape back in
the right-to-left direction to update the encoding accordingly."

We really only need to make a few changes to this to make $M'$ into an
oblivious TM $M''$:

• When $M'$ updates the encoding in its right-to-left sweep, it may need
to move the \^ marker to the right. However, $M''$ cannot stop and move
right when it finds the hat, because it is oblivious. Therefore, the head
of $M''$ must, on its right-to-left sweep, move LRL every time $M'$ would
move L, so its movement pattern will look like LRLLRLLRL... That
way, the head always has the opportunity to move the \^ marker either
left or right when it is encountered.

• We need to know how far out $M''$ needs to sweep. One option is to
calculate $T(n)$ in advance and place a special marker at spot $kT(n)+1$,
so that we always sweep out to $T (n) + 1$ and back to the beginning.
Alternatively, we simply place a marker at spot $k + 1$ in step $1$, and at
the end of every sweep right, we have $M''$ move the marker $k$ cells to
the right. This will work because the original machine $M$ cannot have
gone beyond spot $i$ in step $i$ of its calculation.

• We also need to handle the case where $M$ might halt at different times
depending on the input. We know that $M$ is computable in time $T (n)$
for some time constructible $T$ . This means that there exists a machine
$M_T$ such that on all inputs of length $n$, $M_T$ halts in exactly $T (n)$
steps. We can simulate $M$ and $M_T$ simultaneously on the same input
by adding $m$ tapes (this includes input/output tapes) for simulation
of $M_T$ to $M''$. States of the machine $M''$ can then be considered as
pairs of states of the above form corresponding to $M$ and $M_T$. $M'$ will
continue sweeping back and forth in the oblivious manner described
above until $M_T$ halts.
Because we’re sweeping out to a maximum distance of $kT (n)$, and we do
it $T (n)$ times, the machine $M''$ runs in time $O(T (n)2)$.
\end{latin}


%%%%%%%%%%%%%%%%%%%%%%%%%%%%%%%%%%%
%%%%%%%%%%%%%%%%%%%%%%%%%%%%%%%%%%%
%%%%%%%%%%%%%%%%%%%%%%%%%%%%%%%%%%%



\section*{منابع}
\renewcommand{\section}[2]{}%
\begin{thebibliography}{99} % assumes less than 100 references
%چنانچه مرجع فارسی نیز داشته باشید باید دستور فوق را فعال کنید و مراجع فارسی خود را بعد از این دستور وارد کنید


\begin{LTRitems}

\resetlatinfont

\bibitem{b1} https://www.shiksha.com/online-courses/articles/relu-and-sigmoid-activation-function/
\bibitem{b1} https://medium.com/@amanatulla1606/vanishing-gradient-problem-in-deep-learning-understanding-intuition-and-solutions-da90ef4ecb54
\bibitem{b1} https://en.wikipedia.org/wiki/Rectifier\_(neural\_networks)
\bibitem{b1} https://wandb.ai/ayush-thakur/dl-question-bank/reports/ReLU-vs-Sigmoid-Function-in-Deep-Neural-Networks--VmlldzoyMDk0MzI
\bibitem{b1} https://medium.com/swlh/why-are-neural-nets-non-linear-a46756c2d67f
\end{LTRitems}

\end{thebibliography}


\end{document}

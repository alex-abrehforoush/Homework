\documentclass{article}

\usepackage{graphicx}
\usepackage{fancyhdr}
\usepackage[sorting=none]{biblatex}
\usepackage[margin=1in]{geometry}
\usepackage{listings}
\usepackage{float}
\usepackage{hyperref}
\usepackage{xepersian}

\addbibresource{bibliography.bib}
\settextfont[Scale=1.2]{B-NAZANIN.TTF}
\setlatintextfont[Scale=1]{Times New Roman}
\renewcommand{\baselinestretch}{1.5}
\pagestyle{fancy}
\fancyhf{}
\rhead{تکلیف اول درس سیستم‌های عامل 1 }
\lhead{\thepage}
\rfoot{علیرضا ابره فروش}
\lfoot{9816603}
\renewcommand{\headrulewidth}{1pt}
\renewcommand{\footrulewidth}{1pt}

%%%%%%%%%%%%%%%%
\setcounter{secnumdepth}{3}
\setcounter{tocdepth}{3}
%%%%%%%%%%%%%%%%
\begin{document}
\begin{titlepage}
\begin{center}
\includegraphics[width=0.4\textwidth]{IUT Logo.png}\\
        
\LARGE
\textbf{دانشگاه صنعتی اصفهان}\\
\textbf{دانشکده مهندسی برق و کامپیوتر}\\
        
\vfill
        
\huge
\textbf{عنوان: تکلیف اول درس سیستم‌های عامل 1}\\
        
\vfill
        
\LARGE
\textbf{نام و نام خانوادگی: علیرضا ابره فروش}\\
\textbf{شماره دانشجویی: 9816603}\\
\textbf{نیم\,سال تحصیلی: پاییز 1400}\\
\textbf{مدرّس: دکتر محمّدرضا حیدرپور}\\
\textbf{دستیاران آموزشی: مجید فرهادی - دانیال مهرآیین - محمّد نعیمی}\\
\end{center}
\end{titlepage}


\tableofcontents
\newpage

\section{سوال اول}

\subsection{بخش آ}
هر دو در قسمت \lr{Kenel Space} در \lr{Kenel Stack of Process \#i} هستند. 
\lr{tp} به قسمت \lr{User Space} در کرنل استک اشاره میکند که در آن رجیسترهای \lr{CPU} که متناظر با آن فرایند بوده‌اند، وقتی که فرآیند در \lr{User Space Mode} اجرا می‌شده را وقتی \lr{Interrupt} یا \lr{System Call} یا  \lr{Extension}ی رخ دهد، در اینجا ذخیره می‌کند. در کرنل استک اشاره می‌کند، رجیسترهای \lr{CPU} در آن می‌توانند ذخیره شوند که وقتی در \lr{Kernel Mode} هستیم و \lr{System Call}ی انجام داده‌ایم و \lr{OS} از این استک برای انجام کارش استفاده می‌کند، ممکن است آن موقع \lr{Interrupt} بیاید و یا \lr{Timer} منقضی شود و \lr{Scheduler} تصمیم می‌گیرد که این فرآیند کنار گذاشته شود و فرآیند دیگر باید جایگزین آن شود. بنابراین وضعیت مربوط به \lr{CPU} وقتی که در \lr{Kernel Mode} هستیم، در داخل کرنل قسمت آن فرآیند ذخیره می‌شود.
\subsection{بخش ب}
\lr{User Stack} حاوی اطلاعات داخلی خود پروسس‌ها شامل توابع و سایر موارد می‌باشد. اما در \lr{Kenel Stack} اطلاعات حیاتی سیستم مانند وضعیت پروسس‌ها، پردازنده و رجیسترها ذخیره می‌شود. در نتیجه در حالت گفته شده امنیت سیستم در خطر قرار می‌گیرد و هر پروسس می‌تواند دستورات خاص(\lr{privilege}) که می‌تواند آسیب جدی به سیستم برساند را اجرا کند. پس این حالت بسیاری از مفاهیم امنیتی سیستم عامل زیر سوال می‌رود.
\subsection{بخش ج}
\lr{trap table} همان جدول وقفه هست و مثلا برای اینکه از \lr{User Mode} به \lr{Kernel Mode} منتقل شویم و عمل \lr{trap} به طرف سیستم عامل رخ دهد، از دستور \lr{int} استفاده می‌شود که به صورت آرگومان در \lr{x86} در جدول وقفه، ایندکس 64 آن متناظر با \lr{System Call} هست در واقع چون یک \lr{System Call} است، ایندکس شماره 64 آن صدا زده می‌شود پس برای انتقال از حالت \lr{User} به \lr{Kernel} از آن استفاده می‌شود. 
\lr{syscall-table} هم که نشان می‌دهد جنس \lr{System Call} چه میباشد. ما \lr{System Call} های متفاوتی داریم. اینکه داخل \lr{System Call}، نهایتا \lr{Handler} آن اجرا شود، باید بفهمد که آن چه \lr{System Call}ی بوده است. مثلا در \lr{x86}، عدد 6 که در رجیستر \lr{eax} ریخته می‌شود، نشان می‌دهد جنس سیستم کالر \lr{read} بوده است.
\subsection{بخش د}
محتوای برخی از رجیستر‌ها مانند \lr{Program Counter} یا \lr{Stack Pointer} توسط \lr{kernel handler} قابل ذخیره‌سازی نیستند. چون خود آن‌ها هم نرم‌افزار هستند و تا \lr{CPU} بخواهد آن‌ها را وارد مرحله اجرا کند، محتوای \lr{Program Counter} و \lr{Stack Pointer} عوض می‌شود. پس سخت افزار قبل از فراخوانی  \lr{kernel handler}، به طور خودکار محتوای رجیستر‌های برخی از پروسس‌های متوقف شد را با \lr{push} کردن آن‌ها در \lr{interrup stack} حفظ می‌کند.
\subsection{بخش ه}
امکان آن در الگوریتم‌های \lr{FIFO} و \lr{MLFQ} وجود دارد. در حالت اولیه‌ی \lr{FIFO} اگر يك برنامه با حجم زماني بالا اول وارد صف شود، تا زماني كه تمام نشود نوبت به بقيه نمي‌رسد و در اين حالت براي ديگر برنامه‌ها، \lr{Starvation} رخ مي‌دهد. اگر تعداد \lr{job}های تعاملي زياد باشد و در اولويت اول قرار بگيرند، نوبت به \lr{job}های با اولويت پايين كه حجم بالاتری دارند، نمي‌رسد و دچار \lr{Starvation} مي‌شوند.
\subsection{بخش و}
برای جلوگیری از \lr{Gaming}، قانونی تحت عنوان \lr{Anti-Gaming} ارائه می‌دهیم. قانون به این شکل است که اگر يك \lr{job} زمانی كه برايش تخصيص داده شده را كه با توجه به \lr{Level}ش تنظيم شده، مصرف كند بي‌توجه به فاكتور هاي ديگر، \lr{Level}ش يك واحد افت مي‌كند. در اين صورت برنامه‌هاي تعاملي كه يك اسلايس را تا انتها مصرف نمي‌كنند، نمي‌توانند  \lr{CPU} را \lr{Monopoly} کند.
\subsection{بخش ز}
در الگوريتم \lr{Round Robin} اگر \lr{Time Slice}ها كوچك باشد، \lr{Response Time} کاهش مي‌يابد و اين امر براي كاربران رضايت بخش خواهد بود. اما از طرفي \lr{Over Head} كپي و جابجايي اطلاعات کش‌ها و  رجیسترها كه در بعضي از موارد نيز مورد نیاز است دوباره به \lr{Hard Disk} مراجعه شود، بازدهي را كاهش ميدهد. اگر  \lr{Time Slice}ها بزرگ باشد \lr{job} زمان زيادي را در صف منتظر مي‌ماند و اين عيب محسوب مي‌شود. در نهايت بايد يك تصميم با توجه به شرايط و خواسته‌ها اتخاذ كرد.

\section{سوال دوم}
\indent
سیستم کال \lr{fork()} یک پروسس جدید به نام پروسس \lr{child} می‌سازد. پس از ایجاد پروسس \lr{child}، هر دو پروسس(\lr{child} و \lr{parent}ش)دستوراتی که پس از \lr{fork()} متناظرشان آمده اند را یک به یک اجرا می‌کنند. به این ترتیب پس از اولین \lr{fork()}، در مجموع 2 پروسس، پس از دومین \lr{fork()}، در مجموع 4 پروسس، پس از سومین \lr{fork()}، در مجموع هشت پروسس و $\ldots$، خواهیم داشت. درنتیجه پس از اجرای حلقه، تعداد \lr{child}های ایجاد شده برابر است با:

\begin{center}
$
2+4+\ldots+2^{\log_2 n}=2^{(\log_2 n)+1}=2n
$
\end{center}
در مجموع با احتساب پروسس \lr{parent} اصلی، \lr{n}2 پروسس خواهیم داشت.

\section{سوال سوم}
\subsection{بخش آ}
\begin{figure}[H]
    \centering
    \includegraphics[width=1\textwidth]{figures/Template-Q4-1.png}
    \caption{}
    \label{fig:fig1}
\end{figure}
\subsection{بخش ب}
\begin{figure}[H]
    \centering
    \includegraphics[width=1\textwidth]{figures/Template-Q4-2.png}
    \caption{}
    \label{fig:fig1}
\end{figure}
\section{عنوان سوال چهارم}
یک راه این است که مقدار رجیستری که به جدول وقفه(\lr{Interrupt Vector Table}) اشاره می‌کند را عوض کنیم. درواقع می‌توانیم مکان دلخواهی از حافظه را به عنوان مکان جدول وقفه مشخص کنیم. توسط دستور \lr{lidt} در \lr{x86} می‌توان این رجیستر را مقداردهی کرد. بنابراین این دستور است که مشخص می‌کند که این جدول کجای حافظه قرار گرفته است. بدیهی است که تنها سیستم عامل باید دسترسیِ استفاده از این دستور را داشته باشد. در غیراینصورت، اگر پروسسی بتواند از این دستور استفاده کند، می‌تواند هر کاری را روی سیستم انجام دهد. چون می‌تواند جدول وقفه را طبق نظر خودش تنظیم کند و سپس آدرس حافظه‌ای که به ابتدای آن جدول اشاره می‌کند را در داخل آن رجیستر خاص توسط دستور \lr{lidt} ثبت کند و در نتیجه بسیاری از مفاهیم و سرویس‌هایی که در مورد سیستم عامل مد نظر داشتیم نقض می‌شود.
\lr{\lstinputlisting[language=C, showstringspaces=false, basicstyle=\ttfamily]{sources/lidt.c}}
\begin{figure}[H]
    \centering
    \includegraphics[width=0.8\textwidth]{figures/4.png}
    \caption{خطای رخ داده پس از اجرای برنامه}
    \label{fig:fig1}
\end{figure}
با اجرای برنامه بالا مشاهده می‌شود که \lr{Segmentation fault (core dumped)} رخ می‌دهد. \lr{segmentation fault} یک خطا است که توسط سخت‌افزار در راستای \lr{memory protection}، به سیستم عامل اطلاع می‌دهد که یک برنامه تلاش کرده است که به یک بخش حفاظت شده از حافظه دسترسی پیدا کند(\lr{a memory access violation}). در کامپیوترهای استاندارد \lr{x86}، این یک فرم از \lr{general protection fault} است.

\section{سوال پنجم}
\subsection{}
\lr{\lstinputlisting[language=C, showstringspaces=false, basicstyle=\ttfamily]{sources/collatz_conjecture.c}}
\subsection{}
زیرا در \lr{UNIX/POSIX}، \lr{exit code}ِ یک برنامه از نوع \lr{unsigned 8-bit} تعریف شده است. به طور دقیق‌تر \lr{system call}های خانواده \lr{wait} در \lr{UNIX} نتیجه یک پروسس را به یک \lr{32-bit integer} کدگذاری می‌کنند. از این 32 بیت برای اطلاعاتی همچون وقوع یا عدم وقوع \lr{dump core} در پروسس، \lr{exit} به دلیل یک سیگنال(و چه سیگنالی)، و $\ldots$ تقسیم می‌شود. از این رو تنها 8 بیت از 32 بیت برای \lr{exit code} باقی می‌ماند. پس در نتیجه مقادیر 0 تا 255 به این شکل قابل برگردانده شدن هستند.
\subsection{}
\lr{\lstinputlisting[language=C, showstringspaces=false, basicstyle=\ttfamily]{sources/collatz_conjecture_2.c}}

\section{سوال ششم}
\subsection{بخش آ}
\textbf{\lr{orphan process}:}
پروسسی که \lr{parent}ش وجود ندارد(یا پایان یافته یا بدون اینکه برای متوقف شدن \lr{child}ش صبر کرده باشد، متوقف شده باشد)، \lr{orphan process} نامیده می‌شود.
\newline
\textbf{\lr{zombie process}:}
پروسسی که اجرای آن پایان یافته است اما هنوز در جدول پروسس‌ها مقداری دارد که به پروسس \lr{parent} نسبت داده می‌شود(در جدول پروسس‌ها ورودی دارد)، \lr{zombie process} نامیده می‌شود. یک پروسس \lr{child} همواره پیش از پاک شدن از جدول پروسس‌ها به یک \lr{zombie process} تبدیل می‌شود. پروسس \lr{parent}، \lr{exit status} پروسس \lr{child} را می‌خواند و پروسس \lr{child} از جدول پروسس‌ها حذف می‌شود.
\subsection{بخش ب}
\indent
برنامه‌ی 1 یک \lr{orphan process} ایجاد می‌کند. چون پروسس \lr{child} حدودا 101 ثانیه پس از پایان یافتن پروسس \lr{parent} به پایان می‌رسد. همینطور که در تصویر اول مشهود است، مقدار \lr{ppidِ}ِ پروسس \lr{child} پس از ثانیه اول برابر 2174(\lr{pidِ}ِ پروسس \lr{parent}) است اما پس از ثانیه سوم(پایان یافتن پروسس \lr{parent}) این مقدار برابر 1286(\lr{pid}ِ پروسس \lr{systemd} که نقش \lr{subreaper} را برای پروسس \lr{orphan} برعهده دارد و در قسمت بالای جدول پروسس‌ها قرار دارد) می‌باشد.
\begin{figure}[H]
    \centering
    \includegraphics[width=0.8\textwidth]{figures/6.2.1.1.png}
    \caption{اجرای برنامه‌ی 1}
    \label{fig:fig1}
\end{figure}

\begin{figure}[H]
    \centering
    \includegraphics[width=0.8\textwidth]{figures/6.2.1.3.png}
    \caption{جدول پروسس‌ها(قسمت بالا)}
    \label{fig:fig1}
\end{figure}

\begin{figure}[H]
    \centering
    \includegraphics[width=0.8\textwidth]{figures/6.2.1.2.png}
    \caption{جدول پروسس‌ها(قسمت پایین)}
    \label{fig:fig1}
\end{figure}
%%%%%%%%%%%%%

برنامه‌ی 2 یک \lr{zombie process} ایجاد می‌کند. چون پروسس \lr{parent} حدودا 99 ثانیه پس از پایان یافتن پروسس \lr{child} به پایان می‌رسد. اگر در این بازه زمانی جدول پروسس‌ها را بررسی کنیم می‌بینیم که پروسسی با \lr{STAT}ِ \lr{Z} وجود دارد که \lr{pid}ش برابر \lr{pid}ِ پروسس \lr{child} است و این نشان‌دهنده این است که برنامه‌ی 2 یک \lr{zombie process} تولید کرده است. به تصاویر زیر توجه کنید:
\begin{figure}[H]
    \centering
    \includegraphics[width=0.8\textwidth]{figures/6.2.2.1.png}
    \caption{اجرای برنامه‌ی 2}
    \label{fig:fig1}
\end{figure}

\begin{figure}[H]
    \centering
    \includegraphics[width=0.8\textwidth]{figures/6.2.2.2.png}
    \caption{جدول پروسس‌ها}
    \label{fig:fig1}
\end{figure}

\section{سوال هفتم}
\subsection{سوال 6 فایل}
با اجرای دستور سوال، یک پروسه دارای 3 دستور که از \lr{IOs} و سه پروسه هر کدام دارای 5 دستور که از \lr{CPU} استفاده می‌کنند برای اجرا آماده می‌شوند.
\newline
خیر-منابع به طور موثر و بهینه به کار گرفته نشده‌اند. همانطور که در تصویر اول می‌بینیم \lr{IOs} در بازه‌ی زمانی 7 تا 18 و همچنین \lr{CPU} در بازه‌ی زمانی 19 تا 23 و 26 تا 30 بی‌کار هستند و درنهایت در 31 واحد زمانی \lr{CPU} تنها 67/74 درصد مواقع، و \lr{IOs} تنها 48/39 درصد مواقع مشغول بوده‌اند. به عبارت دیگر در 31 واحد زمانی، \lr{CPU}، 21 واحد زمانی و \lr{IOs}، 15 واحد زمانی درحال استفاده بوده‌اند.
\begin{figure}[H]
    \centering
    \includegraphics[width=0.8\textwidth]{figures/7.1.png}
    \caption{}
    \label{fig:fig1}
\end{figure}
\subsection{سوال 7 فایل}
تفاوتی که این حالت با حالت قبل دارد این است که عملکرد \lr{CPU} و \lr{IOs} در بازه‌های بیشتری همزمان به موازات یکدیگر رخ می‌دهند. درواقع در این حالت پس از پایان یافتن \lr{IO}، بلافاصله به این پروسس سوئیچ می‌کند و درنتیجه 5 دستور اجرای \lr{IO}(\lr{waiting})، با 5 دستور اجرای \lr{CPU} موازی می‌شود. با دقت در تصویر دوم می‌بینیم که زمان کل به 21 واحد کاهش یافته است، \lr{CPU} در کلِ 21 واحد مشغول بوده است، و \lr{IOs} نیز در 15 واحد درحال استفاده بوده است(درست است که در هر دو حالت، \lr{IOs} در 15 واحد زمانی استفاده شده است، اما در حالت قبل، \lr{IOs} تنها در 48/39 درصد مواقع مشغول بوده است، درحالی که در این حالت در 71/43 درصد مواقع درحال استفاده بوده است. پس در این سوال با شرایط ذکر موجود، اجرای بلافصل پروسسی که اخیرا \lr{I/O}ی آن کامل شده است ایده خوبی می‌تواند باشد.
\begin{figure}[H]
    \centering
    \includegraphics[width=0.8\textwidth]{figures/7.2.png}
    \caption{}
    \label{fig:fig1}
\end{figure}


\section*{منابع}
\renewcommand{\section}[2]{}%
\begin{thebibliography}{99} % assumes less than 100 references
%چنانچه مرجع فارسی نیز داشته باشید باید دستور فوق را فعال کنید و مراجع فارسی خود را بعد از این دستور وارد کنید


\begin{LTRitems}

\resetlatinfont

\bibitem{b1}https://www.geeksforgeeks.org/zombie-and-orphan-processes-in-c/

\end{LTRitems}

\end{thebibliography}


\end{document}
